\documentclass[12pt]{article}
\usepackage{lecture}
\usepackage{html}
\usepackage{url}

\newcommand{\copyrightYears}{2001-2021}

\title{Genetic Drift}

\begin{document}

\maketitle

\thispagestyle{first}

\section*{Introduction}

So far in this course we've talked about changes in genotype and
allele frequencies as if they were completely deterministic. Given the
current allele frequencies and viabilities, for example, we wrote down
an equation describing how they will change from one generation to the
next:
\[
p' = \frac{p^2w_{11} + pqw_{12}}{\bar w} \quad .
\]
Notice that in writing this equation, we're claiming that we can
predict the allele frequency in the next generation {\it without
error}. But suppose the population is small, say 10 diploid
individuals, and our prediction is that $p' = 0.5$. Then just as we
wouldn't be surprised if we flipped a coin 20 times and got 12 heads,
we shouldn't be surprised if we found that $p' = 0.6$. The difference
between what we expect ($p' = 0.5$) and what we observe ($p' = 0.6$)
can be chalked up to statistical sampling error. That sampling error
is the cause of~(or just another name for) {\it genetic
drift}{\dash}the tendency for allele frequencies to change from one
generation to the next in a finite population even if there is no
selection.\index{genetic drift}

\section*{A simple example}

To understand in more detail what happens when there is genetic drift,
let's consider the simplest possible example: a haploid population
consisting of 2 individuals.\footnote{Notice that once we start
  talking about genetic drift, we have to specify the size of the
  population. As we'll see, that's because the properties of drift
  depend on how big the population is. We'll also see that the size of
  the population isn't simply the number of individuals we can count.}
Suppose that we are studying a locus with only two alleles in this
population $A_1$ and $A_2$. This implies that $p = q = 0.5$, but we'll
ignore that numerical fact for now and simply label the frequency of
the $A_1$ allele as $p$.

We imagine the following scenario:

\begin{itemize}

\item Each individual in the population produces a very large number
  of haploid gametes that develop directly into adult offspring.

\item The allele in each offspring is an identical copy of the allele
  in its parent, i.e., $A_1$ begets $A_1$ and $A_2$ begets $A_2$. In
  other words, there's no mutation.

\item The next generation is constructed by picking two offspring at
  random from the very large number of offspring produced by these two
  individuals.

\end{itemize}

\noindent Then it's not too hard to see that
\begin{eqnarray*}
\mbox{Probability that both offspring are $A_1$} &=& p^2 \\
\mbox{Probability that one offspring is $A_1$ and one is $A_2$} &=& 2pq \\
\mbox{Probability that both offspring are $A_2$} &=& q^2
\end{eqnarray*}
Of course $p' = 1$ if both offspring sampled are $A_1$, $p' = 1/2$ if
one is $A_1$ and one is $A_2$, and $p' = 0$ if both are $A_2$, so that
set of equations is equivalent to this one:
\begin{eqnarray}
P(p'=1) &=& p^2  \label{eq:p-1} \\
P(p'=1/2) &=& 2pq \\
P(p'=0) &=& q^2  \label{eq:p-2}
\end{eqnarray}
In other words, we can no longer predict with certainty what allele
frequencies in the next generation will be. We can only assign
probabilities to each of the three possible outcomes. Of course, in a
larger population the amount of uncertainty about the allele
frequencies will be smaller,\footnote{More about that later.} but
there will be {\it some\/} uncertainty associated with the predicted
allele frequencies unless the population is infinite.\index{genetic drift!uncertainty in allele frequencies}

The probability of ending up in any of the three possible states
obviously depends on the current allele frequency. In probability
theory we express this dependence by writing equations
(\ref{eq:p-1})--(\ref{eq:p-2}) as conditional
probabilities:
\begin{eqnarray}
P(p_1=1|p_0) &=& p_0^2  \label{eq:p-1-1} \\
P(p_1=1/2|p_0) &=& 2p_0q_0 \\
P(p_1=0|p_0) &=& q_0^2  \label{eq:p-1-2}
\end{eqnarray}
I've introduced the subscripts so that we can distinguish among
various generations in the process. Why? Because if we can write
equations (\ref{eq:p-1-1})--(\ref{eq:p-1-2}), we can also write the
following equations:\footnote{I know. I'm weird. I actually get a kick
  out of writing equations!}
\begin{eqnarray*}
P(p_2=1|p_1) &=& p_1^2 \\
P(p_2=1/2|p_1) &=& 2p_1q_1 \\
P(p_2=0|p_1) &=& q_1^2
\end{eqnarray*}

Now if we stare at those a little while, we\footnote{Or at least the
weird ones among us} begin to see some interesting
possibilities. Namely,

\begin{eqnarray*}
P(p_2=1|p_0) &=& P(p_2=1|p_1=1)P(p_1=1|p_0) + P(p_2=1|p_1=1/2)P(p_1=1/2|p_0) \\
             &=& (1)(p_0^2) + (1/4)(2p_0q_0) \\
             &=& p_0^2 + (1/2)p_0q_0 \\
P(p_2=1/2|p_0) &=& P(p_2=1/2|p_1=1/2)P(p_1=1/2|p_0) \\
               &=& (1/2)(2p_0q_0) \\
               &=& p_0q_0 \\
P(p_2=0|p_0) &=& P(p_2=0|p_1=0)P(p_1=0|p_0) + P(p_2=0|p_1=1/2)P(p_1=1/2|p_0) \\
             &=& (1)(q_0^2) + (1/4)(2p_0q_0) \\
             &=& q_0^2 + (1/2)p_0q_0
\end{eqnarray*}
It takes more algebra than I care to show,\footnote{Ask me, if you're
  really interested.} but these equations can be extended to an
arbitrary number of generations.
\begin{eqnarray*}
P(p_t=1|p_0) &=& p_0^2 + \left(1 - (1/2)^{t-1}\right)p_0q_0 \\
P(p_t=1/2|p_0) &=& p_0q_0(1/2)^{t-2} \\
P(p_t=0|p_0) &=& q_0^2 + \left(1 - (1/2)^{t-1}\right)p_0q_0
\end{eqnarray*}

Why do I bother to show you these equations?\footnote{It's not just
that I'm crazy.} Because you can see pretty quickly that as $t$ gets
big, i.e., the longer our population evolves, the smaller the
probability that $p_t = 1/2$ becomes. In fact, it's not hard to verify
two facts about genetic drift in this simple situation:\index{genetic drift!properties}

\begin{enumerate}

\item One of the two alleles originally present in the population is
certain to be lost eventually.

\item The probability that $A_1$ is fixed is equal to its initial
frequency, $p_0$, and the probability that $A_2$ is fixed is equal to
its initial frequency, $q_0$.

\end{enumerate}

Both of these properties are true in general for {\it any\/} finite
population and {\it any\/} number of alleles.

\begin{enumerate}

\item Genetic drift will eventually lead to loss of all alleles in the
population except one.\footnote{You obviously can't lose all of them
unless the population becomes extinct.}

\item The probability that any allele will eventually become fixed in
the population is equal to its current frequency.

\end{enumerate}

\section*{General properties of genetic drift}

What I've shown you so far applies only to a haploid population with
two individuals. Even I will admit that it isn't a very interesting
situation. Suppose, however, we now consider a populaton with $N$
diploid individuals. We can treat it as if it were a population of
$2N$ haploid individuals using a direct analogy to the process I
described earlier, and then things start to get a little more
interesting.

\begin{itemize}

\item Each individual in the population produces a large number of
gametes.

\item The allele in each gamete is an identical copy of the allele in
  the individual that produced it, i.e., $A_1$ begets $A_1$ and $A_2$
  begets $A_2$.

\item The next generation is constructed by picking $2N$ gametes at
  random from the large number originally produced.

\end{itemize}

We can then write a general expression for how allele frequencies will
change between generations. Specifically, the distribution
describing the probability that there will be $j$ copies of $A_1$ in
the next generation given that there are $i$ copies in this generation
is
\[
P(\hbox{$j$ $A_1$ in offspring $|$ $i$ $A_1$ in parents}) =
{2N \choose j}\left(\frac{i}{2N}\right)\left(1 - \frac{i}{2N}\right)
\quad ,
\]
i.e., a binomial distribution.\index{genetic drift!binomial
  distribution} I'll be astonished if any of what I'm about to say is
apparent to any of you from looking at this equation, but it
implies three really important things. We've encountered the first two
of them already:\index{genetic drift!properties}

\begin{itemize}

\item Allele frequencies will tend to change from one generation to
the next purely as a result of sampling error. As a consequence,
genetic drift will eventually lead to loss of all alleles in the
population except one.

\item The probability that any allele will eventually become fixed in
the population is equal to its current frequency.

\item The population has no memory.\footnote{Technically, we've
    described a Markov chain with a finite state space, but I doubt
    that you really care about that. All Markov chains have this
    ``memoryless'' property. In fact, it's called the Markov
    property~(\url{https://en.wikipedia.org/wiki/Markov_property}).}\index{genetic drift!Markov property} The probability that the 
  offspring generation will have a particular allele frequency depends
  {\it only\/} on the allele frequency in the parental generation. It
  does not depend on how the parental generation came to have that
  allele frequency. This is exactly analogous to coin-tossing. The
  probability that you get a heads on the next toss of a fair coin is
  1/2. It doesn't matter whether you've never tossed it before or if
  you've just tossed 25 heads in a row.\footnote{Of course, if you've
    just tossed 25 heads in a row, you could be forgiven for having
    your doubts about whether the coin is actually fair.}

\end{itemize}

\subsection*{Variance of allele frequencies between
  generations}\index{genetic drift!allele frequency variance}

For a binomial distribution
\begin{latexonly}
\begin{eqnarray*}
P(K=k) &=& {{N \choose k}p^k(1-p)^{N-k}} \\
\hbox{Var}(K) &=& Np(1-p) \\
\hbox{Var}(p) &=& \hbox{Var}(K/N) \\
              &=& \frac{1}{N^2}\hbox{Var}(K) \\
              &=& \frac{p(1-p)}{N}
\end{eqnarray*}
\end{latexonly}
\begin{htmlonly}
\[
P(K=k) = {{N \choose k}p^k(1-p)^{N-k}}
\]
\[
\hbox{Var}(K) = Np(1-p)
\]
\[
\hbox{Var}(p) = \hbox{Var}(K/N)
\]
\[
              = \frac{1}{N^2}\hbox{Var}(K)
\]
\[
              = \frac{p(1-p)}{N}
\]
\end{htmlonly}
Applying this to our situation,
\[
\hbox{Var}(p_{t+1}) = \frac{p_t(1-p_t)}{2N}
\]
Var$(p_{t+1})$ measures the amount of uncertainty about allele
frequencies in the next generation, given the current allele
frequency. As you probably guessed long ago, the amount of uncertainty
is inversely proportional to population size. The larger the
population, the smaller the uncertainty.

If you think about this a bit, you might expect that a smaller
variance would ``slow down'' the process of genetic drift{\dash}and
you'd be right. It takes some pretty advanced mathematics to say how
much the process slows down as a function of population
size,\footnote{Actually, we'll encounter a way that isn't quite so
  hard in a few lectures when we get to the coalescent.} but we can
summarize the result in the following equation:\index{genetic drift!fixation time}
\[
\bar t \approx -4N\left(p\log p + (1-p)\log(1-p)\right) \quad ,
\]
where $\bar t$ is the average time to fixation of one allele or the
other and $p$ is the current allele frequency.\footnote{Notice that
  this equation only applies to the case of one-locus with two
  alleles, although the principle applies to any number of alleles.}
So the average time to fixation of one allele or the other increases
approximately linearly with increases in the population size.

\subsection*{Analogy to inbreeding}\index{genetic drift!inbreeding analogy}

You may have noticed some similarities between drift and
inbreeding. Specifically, both processes lead to a loss of
heterozygosity and an increase in homozygosity. This analogy leads to
a useful heuristic for helping us to understand the dynamics of
genetic drift.

Remember our old friend $f$, the inbreeding coefficient? I'm going to
re-introduce you to it in the form of the population inbreeding
coefficient, the probability that two alleles chosen at random from a
population are identical by descent. We're going to study how the
population inbreeding coefficient changes from one generation to the
next as a result of reproduction in a finite
population.\footnote{Remember that I use the abbreviation ibd to mean
identical by descent.}

\begin{eqnarray*}
f_{t+1} &=& \mbox{Prob. ibd from preceding generation} \\
        &&  + (\mbox{Prob. not ibd from prec. gen.}) \times (\mbox{Prob. ibd from
          earlier gen.}) \\
   &=& \frac{1}{2N} + \left(1 - \frac{1}{2N}\right)f_t
\end{eqnarray*}
or, in general,
\[
f_{t+1} = 1 - \left(1 - \frac{1}{2N}\right)^t(1-f_0) \quad .
\]

\subsection*{Summary}

There are four characteristics of genetic drift that are particularly
important for you to remember:\index{genetic drift!properties}

\begin{enumerate}

\item Allele frequencies tend to change from one generation to the
next simply as a result of sampling error. We can specify a
probability distribution for the allele frequency in the next
generation, but we cannot predict the actual frequency with certainty.

\item There is no systematic bias to changes in allele frequency. The
allele frequency is as likely to increase from one generation to the
next as it is to decrease.

\item If the process is allowed to continue long enough without input
of new genetic material through migration or mutation, the population
will eventually become fixed for only one of the alleles originally
present.\footnote{This will hold true even if there is strong
selection for keeping alleles in the population. Selection can't
prevent loss of diversity, only slow it down.}

\item The time to fixation of a single allele is directly proportional
to population size, and the amount of uncertainty associated with
allele frequencies from one generation to the next is inversely
related to population size.

\end{enumerate}

\section*{Effective population size}

I didn't make a big point of it, but in our discussion of genetic
drift so far we've assumed everything about populations that we
assumed to derive the Hardy-Weinberg principle, {\it and\/} we've
assumed that:\index{genetic drift!ideal population}

\begin{itemize}

\item We can model drift in a finite population as a result of
  sampling among haploid gametes rather than as a result of sampling
  among diploid genotypes. Since we're dealing with a finite
  population, this effectively means that the two gametes incorporated
  into an individual could have come from the same parent, i.e., some
  amount of self-fertilization can occur when there's random union of
  gametes in a finite, diploid population.

\item Since we're sampling gametes rather than individuals, we're also
  implictly assuming that there aren't separate sexes.\footnote{How
    could there be separate sexes if there can be self-fertilization?}

\item The number of gametes any individual has represented in the next
  generation is a binomial random variable.\footnote{More about this
    later.}

\item The population size is constant.

\end{itemize}

How do we deal with the fact that one or more of these conditions will
be violated in just about any case we're interested in?\footnote{OK,
  OK. They will probably be violated in {\it every\/} case we're
  interested in.} One way would be to develop all the probability
models that incorporate that complexity and try to solve them. That's
nearly impossible, except through computer simulations. Another, and
by far the most common approach, is to come up with a conversion
formula that makes our actual population seem like the ``ideal''
population that we've been studying. That's exactly what {\it
  effective population size\/} is.\index{genetic drift!effective population size}
\begin{quote}
The effective size of a population is the size of an ideal population
that has the same properties with respect to genetic drift as our
actual population does.
\end{quote}
What does that phrase ``same properties with respect to genetic
drift'' mean? Well there are two ways it can be
defined.\footnote{There are actually more than two ways, but we're
  only going to talk about two.}

\subsection*{Variance effective size}\index{genetic drift!variance effective size}

You may remember\footnote{You probably won't, so I'll remind you} that
the variance in allele frequency in an ideal population is
\[
Var(p_{t+1}) = \frac{p_t(1-p_t)}{2N} \quad.
\]
So one way we can make our actual population equivalent to an ideal
population to make their allele frequency variances the same. We do
this by calculating the variance in allele frequency for our actual
population, figuring out what size of ideal population would produce
the same variance, and pretending that our actual population is the
same as an ideal population of the same size. To put that into an
equation,\footnote{As if that will make it any clearer. Does anyone
  actually read these footnotes?} let $\widehat{Var}(p)$ be the
variance we calculate for our actual population. Then\index{variance effective size}
\[
N_e^{(v)} = \frac{p(1-p)}{2\widehat{Var}(p)}
\]
is the {\it variance effective population size}, i.e., the size of an
ideal population that has the same properties with respect to allele
frequency variance as our actual population.

\subsection*{Inbreeding effective size}\index{genetic drift!inbreeding effective size}

You may also remember that we can think of genetic drift as analogous
to inbreeding. The probability of identity by descent within
populations changes in a predictable way in relation to population
size, namely\index{inbreeding effective size}
\[
f_{t+1}
   = \frac{1}{2N} + \left(1 - \frac{1}{2N}\right)f_t \quad.
\]
So another way we can make our actual population equivalent to an
ideal population is to make them equivalent with respect to how $f$
changes from generation to generation. We do this by calculating how
the inbreeding coefficient changes from one generation to the next in
our actual population, figuring out what size an ideal population
would have to be to show the same change between generations, and
pretending that our actual population is the same size at the ideal
one. So suppose $\hat f_t$ and $\hat f_{t+1}$ are the actual
inbreeding coefficients we'd have in our population at generation $t$
and $t+1$, respectively. Then
\begin{eqnarray*}
\hat f_{t+1} &=& \frac{1}{2N_e^{(f)}} + \left(1 -
        \frac{1}{2N_e^{(f)}}\right)\hat f_t \\
        &=& \left(\frac{1}{2N_e^{(f)}}\right)(1 - \hat f_t) + \hat f_t \\
\hat f_{t+1} - \hat f_t &=& \left(\frac{1}{2N_e^{(f)}}\right)(1 - \hat
        f_t) \\
N_e^{(f)} &=& \frac{1 - \hat f_t}{2(\hat f_{t+1} - \hat f_t)} \quad .
\end{eqnarray*}
In many applications it's convenient to assume that $\hat f_t = 0$. In
that case the calculation gets much simpler:
\[
N_e^{(f)} = \frac{1}{2\hat f_{t+1}} \quad .
\]
We also don't lose anything by taking the simpler approach, because
$N_e^{(f)}$ depends only on how much $f$ {\it changes\/} from one
generation to the next, not on its actual magnitude.

\subsection*{Comments on effective population sizes}\index{genetic drift!effective population size, limitations}

Those are nice tricks, but there are some limitations. The biggest is
that $N_e^{(v)} \ne N_e^{(f)}$ if the population size is changing from
one generation to the next.\footnote{It's even worse than that. When
  the population size is changing, it's not clear that any of the
  available adjustments to produce an effective population size are
  entirely satisfactory. Well, that's not entirely true either. Fu et
  al.~\cite{Fu-etal-2003} show that there is a reasonable definition
  in one simple case when the population size varies, and it happens
  to correspond to the solution presented below.} So you have to
decide which of these two measures is more appropriate for the
question you're studying.

\begin{itemize}

\item $N_e^{(f)}$ is naturally related to the number of individuals in
the parental populations. It tells you something about how the
probability of identity by descent within a single population will
change over time.

\item $N_e^{(v)}$ is naturally related to the number of individuals in
the offspring generation. It tells you something about how much allele
frequencies in isolated populations will diverge from one another.

\end{itemize}

\subsection*{Examples}

This is all pretty abstract. Let's work through some examples to see
how this all plays out.\footnote{If you're interested in a
  comprehensive list of formulas relating various demographic
  parameters to effective population size, take a look
  at~\cite[p. 362]{Crow-Kimura-1970}. They provide a pretty
  comprehensive summary and a number of derivations.} In the case of
separate sexes and variable population size, I'll provide a derivation
of $N_e^{(f)}$. In the case of differences in the number of offspring
left by individuals, I'll just give you the formula and we'll discuss
some of the implications.

\subsubsection*{Separate sexes}\index{genetic drift!effective population size, separate sexes}

We'll start by assuming that $\hat f_t = 0$ to make the calculations
simple. So we know that
\[
N_e^{(f)} = \frac{1}{2\hat f_{t+1}} \quad .
\]
The first thing to do is to calculate $\hat f_{t+1}$. To do this we
have to break the problem down into pieces.\footnote{Remembering, of
course, that $\hat f_{t+1}$ is the probability that two alleles drawn
at random are identical by descent.}

\begin{itemize}

\item We assumed that $\hat f_t = 0$, so the only way for two alleles
to be identical by descent is if they are identical copies of the {\it
same\/} allele in the immediately preceding generation.

\item Even if the numbers of reproductive males and reproductive
  females are different, every new offspring has exactly one father
  and one mother. Thus, the probability that the first gamete selected
  at random is female is just 1/2, and the probability that the first
  gamete selected is male is just 1/2.

\item The probability that the second gamete selected is female given
  that the first one we selected was female is $(N-1)/(2N-1)$, because
  $N$ out of the $2N$ alleles represented among offspring came from
  females, and there are only $N-1$ out of $2N-1$ left after we've
  already picked one. The same logic applies for male gametes.

\item The probability that one particular female gamete was chosen is
  $1/2N_f$, where $N_f$ is the number of females in the
  population. Similarly the probability that one particular male
  gamete was chosen is $1/2N_m$, where $N_m$ is the number of males in
  the population.

\end{itemize}

\noindent With those facts in hand, we're ready to calculate $\hat
f_{t+1}$.

\begin{eqnarray*}
f_{t+1} &=& \left(\half\right) \left(\frac{N-1}{2N-1}\right)
            \left(\frac{1}{2N_f}\right) +
            \left(\half\right) \left(\frac{N-1}{2N-1}\right)
            \left(\frac{1}{2N_m}\right) \\
        &=& \left(\half\right) \left(\frac{N-1}{2N-1}\right)
            \left(\frac{1}{2N_f} + \frac{1}{2N_m}\right) \\
        &\approx& \left(\quarter\right)
            \left(\frac{2N_m + 2N_f}{4N_fN_m}\right) \\
        &=& \left(\half\right)
            \left(\frac{N_m + N_f}{4N_fN_m}\right)
\end{eqnarray*}
So,
\[
N_e^{(f)} \approx \frac{4N_fN_m}{N_f + N_m} \quad .
\]

What does this all mean? Well, consider a couple of important
examples. Suppose the numbers of females and males in a population are
equal, $N_f = N_m = N/2$. Then
\begin{eqnarray*}
N_e^{(f)} &=& \frac{4(N/2)(N/2)}{N/2 + N/2} \\
          &=& \frac{4N^2/4}{N} \\
          &=& N \quad .
\end{eqnarray*}
The effective population size is equal to the actual population size
if the sex ratio is 50:50. If it departs from 50:50, the effective
population size will be smaller than the actual population
size.

Consider the extreme case where there's only one reproductive
male in the population. Then
\begin{equation}
N_e^{(f)} = \frac{4N_f}{N_f + 1} \quad . \label{eq:ne-harem}
\end{equation}
Notice what this equation implies: The effective size of a population
with only one reproductive male (or female) can {\it never\/} be
bigger than 4, no matter how many mates that individual has and no
matter how many offspring are produced. At first, this is a little
surprising, but if when you realize that under these conditions all
offspring are half sibs, it may be a little less surprising.

\subsubsection*{Variable population size}\index{genetic drift!effective population size, variable population size}

The notation for this one gets a little more complicated, but the
ideas are simpler than those you just survived. Since the population
size is changing we need to specify the population size at each time
step. Let $N_t$ be the population size in generation $t$. Then
\begin{eqnarray*}
f_{t+1} &=& \left(1-\frac{1}{2N_t}\right)f_t + \frac{1}{2N_t} \\
1 - f_{t+1} &=& \left(1-\frac{1}{2N_t}\right)(1-f_t) \\
1 - f_{t+K} &=&
\left(\prod_{i=1}^K\left(1-\frac{1}{2N_{t+i}}\right)\right)(1-f_t) \quad .
\end{eqnarray*}
Now if the population size were constant
\[
\left(\prod_{i=1}^K\left(1-\frac{1}{2N_{t+i}}\right)\right) =
\left(1 - \frac{1}{2N_e^{(f)}}\right)^K \quad .
\]
Dealing with products and powers is inconvenient, but if we take the
logarithm of both sides of the equation we get something
simpler:\footnote{OK. I know it doesn't look any simpler, but trust me
  it is. We can work with this one. The other one we can only stare
  at.} 
\[
\sum_{i=1}^K\log\left(1-\frac{1}{2N_{t+i}}\right) =
K\log\left(1 - \frac{1}{2N_e^{(f)}}\right) \quad .
\]
It's a well-known fact\footnote{Well known to some of us at least.}
that $\log(1-x) \approx -x$ when $x$ is small. So if we assume that
$N_e$ and all of the $N_{t}$ are large,\footnote{So that their
  reciprocals are small} then
\begin{eqnarray*}
K\left(-\frac{1}{2N_e^{(f)}}\right)
  &=& \sum_{i=1}^K-\frac{1}{2N_{t+i}} \\
\frac{K}{N_e^{(f)}} &=& \sum_{i=1}^K\frac{1}{N_{t+i}} \\
N_e^{(f)} &=& \left(\left(\frac{1}{K}\right)
                    \sum_{i=1}^K\frac{1}{N_{t+i}}\right)^{-1}
\end{eqnarray*}

The quantity on the right side of that last equation is a well-known
quantity. It's the {\it harmonic mean} of the $N_{t}$.\index{harmonic mean}  It's another
well-known fact\footnote{Are we ever going to run out of well-known
  facts? Probably not.} that the harmonic mean of a series of numbers
is always less than its arithmetic mean. This means that genetic drift
may play a much more imporant role than we might have imagined, since
the effective size of a population will be more influenced by times
when it is small than by times when it is large.

Consider, for example, a population in which $N_1$ through $N_9$ are
1000, and $N_{10}$ is 10.
\begin{eqnarray*}
N_e &=& \left(\left(\frac{1}{10}\right)
            \left(9\left(\frac{1}{1000}\right) +
                   \left(\frac{1}{10}\right)\right)\right)^{-1} \\
    &\approx& 92
\end{eqnarray*}
{\it versus\/} an arithmetic average of 901. So the population will
behave with respect to the inbreeding associated with drift like a
population a tenth of its arithmetic average size.

\subsection*{Variation in offspring number}\index{genetic drift!effective population size, variation in offspring number}

I'm just going to give you this formula. I'm not going to derive it
for you.\footnote{The details are in~\cite{Crow-Kimura-1970}, if
  you're interested.}
\[
N_e^{(f)} = \frac{2N - 1}{1 + \frac{V_k}{2}} \quad ,
\]
where $V_k$ is the variance in number of offspring among individuals
in the population.  Remember I told you that the number of gametes any
individual has represented in the next generation is a binomial random
variable in an ideal population? Well, if the population size isn't
changing, that means that $V_k = 2(1 - 1/N)$ in an ideal
population.\footnote{The calculation is really easy, and I'd be happy
  to show it to you if you're interested.} A little algebra should
convince you that in this case $N_e^{(f)} = N$. It can also be shown
(with more algebra) that

\begin{itemize}

\item $N_e^{(f)} < N$ if $V_k > 2(1 - 1/N)$ and

\item $N_e^{(f)} > N$ if $V_k < 2(1 - 1/N)$.

\end{itemize}

\noindent That last fact is pretty remarkable. Conservation biologists
try to take advantage of it to decrease the loss of genetic variation
in small populations, especially those that are captive bred. If you
can reduce the variance in reproductive success, you can substantially
increase the effective size of the population. In fact, if you could
reduce $V_k$ to zero, then
\[
N_e^{(f)} = 2N - 1 \quad .
\]
The effective size of the population would then be almost twice its
actual size.

\bibliography{popgen}
\bibliographystyle{plain}

\ccLicense

\end{document}
