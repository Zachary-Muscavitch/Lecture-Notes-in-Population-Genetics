\chapter{Selection on multiple characters}

So far we've studied only the evolution of a single trait, e.g.,
height or weight. But organisms have many traits, and they evolve at
the same time. How can we understand their simultaneous evolution? The
basic framework of the quantitative genetic approach was first
outlined by Russ Lande and Steve
Arnold~\cite{Lande-Arnold-1983}.\index{response to selection}\index{selection differential}\index{breeders equation}

Let $z_1$, $z_2$, \dots, $z_n$ be the phenotype of each character that
we are studying. We'll use $\bar{\bf z}$ to denote the vector of these
characters before selection and $\bar{\bf z}^*$ to denote the vector after
selection. The selection differential, $\bf s$, is also a vector
given~by
\[
{\bf s} = \bar{\bf z}^* - \bar{\bf z} \quad .
\]
Suppose $p({\bf z})$ is the probability that any individual has
phenotype $\bf z$, and let $W({\bf z})$ be the fitness (absolute
viability) of an individual with phenotype $\bf z$. Then the mean
absolute fitness~is
\[
\bar W = \int W({\bf z})p({\bf z})d{\bf z} \quad .
\]
The relative fitness of phenotype $\bf z$ can be written~as
\[
w({\bf z}) = {W({\bf z}) \over \bar W} \quad .
\]
Using relative fitnesses the mean relative fitness, $\bar w$, is
1. Now
\[
\bar{\bf z}^* = \int {\bf z}w({\bf z})p({\bf z})d{\bf z} \quad .
\]
Recall that $Cov(X,Y) = E(X - \mu_x)(Y - \mu_y) = E(XY) -
\mu_x\mu_y$. Consider 
\begin{eqnarray*}
{\bf s} &=& \bar{\bf z}^* - \bar{\bf z} \\
        &=& \int {\bf z}w({\bf z})p({\bf z})d{\bf z} - \bar {\bf z} \\
        &=& E(w,z) - \bar w\bar {\bf z} \quad ,
\end{eqnarray*}
where the last step follows since $\bar w = 1$ meaning that $\bar
w\bar{\bf z} = \bar{\bf z}$. In short, 
\[
{\bf s} = Cov(w,z) \quad .
\]
That should look familiar from our analysis of the evolution of a
single phenotype.

If we assume that all genetic effects are additive, then the phenotype
of an individual can be written as
\[
{\bf z} = {\bf x} + {\bf e} \quad ,
\]
where $\bf x$ is the additive genotype and $\bf e$ is the
environmental effect. We'll denote by $\bf G$ the matrix of genetic
variances and covariances and by $\bf E$ the matrix of environmental
variances and covariances. The matrix of phenotype variances and
covariances, $\bf P$, is then given by\footnote{Assuming that there
  are no genotype $\times$ environment interactions.}\index{G-matrix@$G$-matrix}\index{P-matrix@$P$-matrix}\index{E-matrix@$E$-matrix}
\[
{\bf P} = {\bf G} + {\bf E} \quad .
\]
Now, if we're willing to assume that the regression of additive
genetic effects on phenotype is linear\footnote{And we were willing to
do this when we were studying the evolution of only one trait, so why
not do it now?} and that the environmental variance is the same for
every genotype, then we can predict how phenotypes will change from
one generation to the next
\begin{eqnarray*}
\bar{\bf x}^* - \bar{\bf x} &=& {\bf GP}^{-1}(\bar{\bf z}^* - \bar{\bf z}) \\
\bar{\bf z}'  - \bar{\bf z} &=& {\bf GP}^{-1}(\bar{\bf z}^* - \bar{\bf z}) \\
\Delta\bar{\bf z} &=& {\bf GP}^{-1}{\bf s}
\end{eqnarray*}
${\bf GP}^{-1}$ is the multivariate version of $h^2_N$. This equation
is also the multivariate version of the breeders
equation.\index{breeders equation}

But we have already seen that ${\bf s} = Cov(w,z)$. Thus, 
\[
{\bf \beta} = {\bf P}^{-1}{\bf s}
\]
is a set of partial regression coefficients of relative fitness on the
characters, i.e., the dependence of relative fitness on that character
alone holding all others constant.

Note:
\begin{eqnarray*}
s_i &=& \sum_{j=1}^n \beta_jP_{ij} \\
    &=& \beta_1P_{i1} + \cdots + \beta_iP_{ii} + \cdots + \beta_nP_{in}
\end{eqnarray*}
is the total selective differential in character $i$, including the
indirect effects of selection on other characters.

\section*{An example: selection in a pentastomid bug}

94 individuals were collected along shoreline of Lake Michigan in
Parker County, Indiana after a storm. 39 were alive, 55 dead. The
means of several characters before selection, the trait correlations,
and the selection analysis are presented in
Table~\ref{table:data}.\index{selection!multivariate example}

\begin{table}
\begin{center}
\begin{tabular}{l|cc}
\hline\hline
Character & Mean before selection & standard deviation \\
\hline
head      & 0.880                 & 0.034 \\
thorax    & 2.038                 & 0.049 \\
scutellum & 1.526                 & 0.057 \\
wing      & 2.337                 & 0.043 \\
\hline
\end{tabular}
\vskip 4pt
\begin{tabular}{l|cccc}
\hline\hline
          & head & thorax & scutellum & wing \\
\hline
head      & 1.00 & 0.72   & 0.50      & 0.60 \\
thorax    &      & 1.00   & 0.59      & 0.71 \\
scutellum &      &        & 1.00      & 0.62 \\
wing      &      &        &           & 1.00 \\
\hline
\end{tabular}
\vskip 4pt
\begin{tabular}{l|cccc}
\hline\hline
Character & $s$    & $s'$  & $\beta$ & $\beta'$ \\
\hline
head      & -0.004 & -0.11 & -0.7 $\pm$ 4.9 & -0.03 $\pm$ 0.17 \\
thorax    & -0.003 & -0.06 & 11.6 $\pm$ 3.9$^{**}$ & 0.58 $\pm$ 0.19$^{**}$
\\
scutellum & -0.16$^*$ & -0.28$^*$ & -2.8 $\pm$ 2.7 & -0.17 $\pm$ 0.15 \\
wing      & -0.019$^{**}$ & -0.43$^{**}$ & -16.6 $\pm$ 4.0$^{**}$ & -0.74 $\pm$
0.18$^{**}$ \\
\hline
\end{tabular}
\end{center}
\caption{Selection analysis of pentastomid bugs on the shores of Lake
Michigan.}\label{table:data}
\end{table}

The column labeled $s$ is the selective differential for each
character. The column labeled $s'$ is the {\it standardized\/}
selective differential, i.e., the change measured in units of standard
deviation rather than on the original scale.\footnote{To measure on
  this scale the data is simply transformed by setting $y_i = (x_i -
  \bar x)/s$, where $x_i$ is the raw score for the $i$th individual,
  $\bar x$ is the sample mean for the trait, and $s$ is its standard
  deviation.} A multiple regression analysis of fitness versus
phenotype on the original scale gives estimates of $\beta$, the direct
effect of selection on that trait. A multiple regression analysis of
fitness versus phenotype on the transformed scale gives the
standardized direct effect of selection, $\beta'$, on that trait.

Notice that the selective differential\footnote{The cumulative effect
  of selection on the change in mean phenotype.} for the thorax
measurement is negative, i.e., individuals that survived had smaller
thoraces than those that died. But the {\it direct\/} effect of
selection on thorax is strongly positive, i.e., all other things being
equal, an individual with a large was more likely to survive than one
with a small thorax. Why the apparent contradiction? Because the
thorax measurement is positively correlated with the wing measurement,
and there's strong selection for decreased values of the wing
measurement.

\section*{Cumulative selection gradients}\index{cumulative selection gradient}

Arnold~\cite{Arnold-1988} suggested an extension of this approach to
longer evolutionary time scales. Specifically, he studied variation in
the number of body vertebrae and the number of tail vertebrae in
populations of {\it Thamnophis elegans} from two regions of central
California. He found relatively little vertebral variation within
populations, but there were considerable differences in vertebral
number between populations on the coast side of the Coast Ranges and
populations on the Central Valley side of the Coast Ranges. The
consistent difference suggested that selection might have produced
these differences, and Arnold attempted to determine the amount of
selection necessary to produce these~differences.

\subsection*{The data}

Arnold collected pregnant females from two local populations in each
of two sites in northern California 282 km apart from one
another. Females were collected over a ten-year period and returned to
the University of Chicago. Dam-offspring regressions were used to
estimate additive genetic variances and covariances of vertebral
number.\footnote{1000 progeny from 100 dams.}  Mark-release-recapture
experiments in the California populations showed that females with
intermediate numbers of vertebrae grow at the fastest rate, at least
at the inland site, although no such relationship was found in
males. The genetic variance-covariance matrix he obtained is shown in
Table~\ref{table:arnold-data}.

\begin{table}
\begin{center}
\begin{tabular}{l|cc}
\hline\hline
     & body    & tail \\
\hline
body & 35.4606 & 11.3530 \\
tail & 11.3530 & 37.2973 \\
\hline
\end{tabular}
\end{center}
\caption{Genetic variance-covariance matrix for vertebral number in
central Californian garter snakes.}\label{table:arnold-data}
\end{table}

\subsection*{The method}

We know from Lande and Arnold's results that the change in
multivariate phenotype from one generation to the next,
$\Delta\bar{\bf z}$, can be written~as
\[
\Delta\bar{\bf z} = {\bf G\beta} \quad ,
\]
where $\bf G$ is the genotypic variance-covariance matrix, ${\bf\beta}
= {\bf P}^{-1}{\bf s}$ is the set of partial regression coefficients
describing the direct effect of each character on relative
fitness.\footnote{{\bf P} is the phenotypic variance-covariance matrix
and {\bf s} is the vector of selection differentials.} If we are
willing to assume that {\bf G} remains constant, then the total change
in a character subject to selection for $n$ generations~is
\[
\sum_{k=1}^n \Delta\bar{\bf z} = {\bf G}\sum_{k=1}^n\beta \quad .
\]
Thus, $\sum_{k=1}^n\beta$ can be regarded as the cumulative selection
differential associated with a particular observed change, and it can
be estimated~as
\[
\sum_{k=1}^n\beta = {\bf G}^{-1}\sum_{k=1}^n \Delta\bar{\bf z}\quad .
\]

\subsection*{The results}

The overall difference in vertebral number between inland and coastal
populations can be summarized~as:
\begin{eqnarray*}
\mbox{body}_{\mbox{inland}} - \mbox{body}_{\mbox{coastal}} &=& 16.21 \\
\mbox{tail}_{\mbox{inland}} - \mbox{tail}_{\mbox{coastal}} &=& 9.69
\end{eqnarray*}
Given the estimate of $\bf G$ already obtained, this corresponds to a
cumulative selection gradient between inland and coastal
populations~of
\begin{eqnarray*}
\beta_{\mbox{body}} &=& 0.414 \\
\beta_{\mbox{tail}} &=& 0.134
\end{eqnarray*}

Applying the same technique to looking at the differences between
populations within the inland site and within the coastal site we find
cumulative selection gradients~of
\begin{eqnarray*}
\beta_{\mbox{body}} &=& 0.035 \\
\beta_{\mbox{tail}} &=& 0.038
\end{eqnarray*}
for the coastal site~and
\begin{eqnarray*}
\beta_{\mbox{body}} &=& 0.035 \\
\beta_{\mbox{tail}} &=& -0.004
\end{eqnarray*}
for the inland site.

\subsection*{The conclusions}

``To account for divergence between inland and coastal California, we
must invoke cumulative forces of selection that are 7 to 11 times
stronger than the forces needed to account for differentiation of
local populations.''

Furthermore, recall that the selection gradients can be used to
partition the overall response to selection in a character into the
portion due to the direct effects of that character alone and the
portion due to the indirect effects of selection on a correlated
character. In this case the overall response to selection in number of
body vertebrae is given~by
\[
{\bf G}_{11}\beta_1 + {\bf G}_{12}\beta_2 \quad ,
\]
where ${\bf G}_{11}\beta_1$ is the direct effect of body vertebral
number and ${\bf G}_{12}\beta_2$ is the indirect effect of tail
vertebral number. Similarly, the overall response to selection in
number of tail vertebrae is given~by
\[
{\bf G}_{12}\beta_1 + {\bf G}_{22}\beta_2 \quad ,
\]
where ${\bf G}_{22}\beta_2$ is the direct effect of tail vertebral
number and ${\bf G}_{12}\beta_1$ is the indirect effect of body
vertebral number. Using these equations it is straightforward to
calculate that 91\% of the total divergence in number of body
vertebrae is a result of direct selection on this character. In
contrast, only 51\% of the total divergence in number of tail
vertebrae is a result of direct selection on this character, i.e.,
49\% of the difference in number of tail vertebrae is attributable to
indirect selection as a result of its correlation with number of
body~vertebrae.

\subsection*{The caveats}\index{cumulative selection gradient!caveats}

While the approach Arnold suggests is intriguing, there are a number
of caveats that must be kept in mind in trying to apply it.

\begin{itemize}

\item This approach assumes that the $\bf G$ matrix remains constant.

\item This approach cannot distinguish strong selection that happened
over a short period of time from weak selection that happened over a
long period of time.

\item This approach {\it assumes\/} that the observed differences in
populations are the result of selection, but populations isolated from
one another will diverge from one another even in the absence of
selection simply as a result of genetic~drift.

\begin{itemize}

\item Small amount of differentiation between populations within sites
could reflect relatively recent divergence of those populations from a
common ancestral~population.

\item Large amount of differentiation between populations from inland
versus coastal sites could reflect a more ancient divergence from a
common ancestral~population.

\end{itemize}

\end{itemize}

