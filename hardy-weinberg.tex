\documentclass[12pt]{article}
\usepackage{lecture}
\usepackage{html}
\usepackage{graphicx}
\usepackage{epstopdf}

\newcommand{\copyrightYears}{2001-2021}

\title{The Hardy-Weinberg Principle and estimating allele frequencies}

\begin{document}

\maketitle

\thispagestyle{first}

\section*{Introduction}

To keep things relatively simple, we'll spend much of our time in the
first part of this course talking about variation at a single genetic
locus, even though alleles at many different loci are involved in
expression of most morphological or physiological traits. Towards the
end of the course, we'll study the genetics of continuous
(quantitative) variation, but until then you can asssume that I'm
talking about variation at a single locus unless I specifically say
otherwise.\footnote{You'll see in a week or a week and a half when we
  talk about analysis of population structure that we start discussing
  variation at many loci. But you'll also see that in spite of
  discussing variation at many loci simultaneously, virtually all of
  the underlying mathematics is based on the properties of those loci
  considered one at a time.}

\section*{The genetic composition of populations}

When I talk about the genetic composition of a population, I'm
referring to three aspects of genetic variation within that
population:\footnote{At each locus I'm talking about. Remember, I'm
  only talking about one locus at a time, unless I specifically say
  otherwise. We'll see why this matters when I outline the ideas
  behind genome-wide association mapping.}\index{genetic composition of populations} 
\begin{enumerate}

\item The number of alleles at a locus.

\item The frequency of alleles at the locus.

\item The frequency of genotypes at the locus.

\end{enumerate}
It may not be immediately obvious why we need both (2) and
(3) to describe the genetic composition of a population, so let me
illustrate with two hypothetical populations:
\begin{center}
\begin{tabular}{lrrr}
             & $A_1A_1$ & $A_1A_2$ & $A_2A_2$ \\
\hline\hline
Population 1 &       50 &        0 &       50 \\
Population 2 &       25 &       50 &       25 \\
\hline
\end{tabular}
\end{center}
It's easy to see that the frequency of $A_1$ is 0.5 in both
populations,\footnote{$p_1 = 2(50)/200 = 0.5$,
  $p_2 = (2(25) + 50)/200 = 0.5$.} but the genotype frequencies are
very different. In point of fact, we don't need both genotype and
allele frequencies. We could get away with only genotype frequencies,
since we can always calculate allele frequencies from genotype
frequencies. But there are fewer allele frequencies than genotype
frequencies{\dash}only one allele frequency when there are two alleles
at a locus. So working with allele frequencies is more convenient when
we can get away with it. The challenge is that we can't get genotype
frequencies from allele frequencies unless $\dots$

\section*{Derivation of the Hardy-Weinberg principle}

We saw last time using the data from {\it Zoarces viviparus\/} that we
can describe empirically and algebraically how genotype frequencies in
one generation are related to genotype frequencies in the next. Let's
explore that a bit further. To do so we're going to use a technique
that is broadly useful in population genetics,\footnote{Although to be
  honest, we won't see mating tables again after the first couple
  weeks of the semester.} i.e., we're going to
construct a mating table. A mating table consists of three
components:\index{mating table}

\begin{enumerate}

\item A list of all possible genotype pairings.

\item The frequency with which each genotype pairing occurs.

\item The genotypes produced by each pairing.

\end{enumerate}

\begin{center}
\begin{tabular}{rcccc}
\hline\hline
                       &           & \multicolumn{3}{c}{Offspring genotype} \\
Female $\times$ Male   & Frequency     & $A_1A_1$ & $A_1A_2$ & $A_2A_2$ \\
\hline
$A_1A_1 \times A_1A_1$ & $x_{11}^2$     &        1 &        0 &        0 \\
              $A_1A_2$ & $x_{11}x_{12}$ &    $\half$ &    $\half$ &        0 \\
              $A_2A_2$ & $x_{11}x_{22}$ &        0 &        1 &        0 \\
$A_1A_2 \times A_1A_1$ & $x_{12}x_{11}$ &    $\half$ &    $\half$ &        0 \\
              $A_1A_2$ & $x_{12}^2$     &  $\fourth$ &    $\half$ &  $\fourth$ \\
              $A_2A_2$ & $x_{12}x_{22}$ &        0 &    $\half$ &    $\half$ \\
$A_2A_2 \times A_1A_1$ & $x_{22}x_{11}$ &        0 &        1 &        0 \\
              $A_1A_2$ & $x_{22}x_{12}$ &        0 &    $\half$ &    $\half$ \\
              $A_2A_2$ & $x_{22}^2$     &        0 &         0 &
                       1 \\
\hline
\end{tabular}
\end{center}
Notice that I've distinguished matings by both maternal and paternal
genotype. While it's not necessary for this example, we will see
examples later in the course where it's important to distinguish a
mating in which the female is $A_1A_1$ and the male is $A_1A_2$ from
ones in which the female is $A_1A_2$ and the male is $A_1A_1$. You are
also likely to be surprised to learn that just in writing this table
we've already made three assumptions about the transmission of genetic
variation from one generation to the next:\index{Hardy-Weinberg
  assumptions}

\begin{description}

\item[Assumption \#1] Genotype frequencies are the same in males and
  females, e.g., $x_{11}$ is the frequency of the $A_1A_1$ genotype in
  both males and females.\footnote{It would be easy enough to relax
    this assumption, but it makes the algebra more complicated without
    providing any new insight, so we won't bother with relaxing it
    unless someone asks.}

\item[Assumption \#2] Genotypes mate at random {\it with respect to
  their genotype at this particular locus}.

\item[Assumption \#3] Meiosis is fair. More specifically, we assume
  that there is no segregation distortion; no gamete competition; no
  differences in the developmental ability of eggs, or the
  fertilization ability of sperm.\footnote{We are also assuming that
    we're looking at offspring genotypes at the zygote stage, so that
    there hasn't been any opportunity for differential survival.} It
  may come as a surprise to you, but there are alleles at some loci in
  some organisms that subvert the Mendelian rules, e.g., the $t$
  allele in house mice, segregation distorter in {\it Drosophila
    melanogaster}, and spore killer in {\it Neurospora
    crassa\/}.\footnote{If you're interested, a pair of papers
    describing work on spore killer in {\it Neurospora\/} appeared in
    2012~\cite{Hammond-etal-2012,Saupe-2012}.}

\end{description}
Now that we have this table we can use it to calculate the frequency
of each genotype in newly formed zygotes in the
population,\footnote{Not just the offspring from these matings.}
provided that we're willing to make three additional assumptions:

\begin{description}

\item[Assumption \#4] There is no input of new genetic material, i.e.,
gametes are produced without mutation, and all offspring are produced
from the union of gametes within this population, i.e., no migration
from outside the population.

\item[Assumption \#5] The population is of infinite size so that the
actual frequency of matings is equal to their expected frequency and
the actual frequency of offspring from each mating is equal to the
Mendelian expectations.

\item[Assumption \#6] All matings produce the same number of
offspring, on average.

\end{description}
Taking these three assumptions together allows us to conclude that the
frequency of a particular genotype in the pool of newly formed zygotes
is
\[
\sum(\hbox{frequency of mating})(\hbox{frequency of genotype produce
  from mating}) \quad .
\]
So

\begin{eqnarray*}
\hbox{freq.}(A_1A_1\hbox{ in zygotes}) &=&
   x_{11}^2 + \frac{1}{2}x_{11}x_{12} + \frac{1}{2}x_{12}x_{11}
   + \frac{1}{4}x_{12}^2 \\
&=& x_{11}^2 + x_{11}x_{12} + \frac{1}{4}x_{12}^2 \\
&=& (x_{11} + x_{12}/2)^2 \\
&=& p^2 \\
\hbox{freq.}(A_1A_2\hbox{ in zygotes}) &=& 2pq \\
\hbox{freq.}(A_2A_2\hbox{ in zygotes}) &=& q^2 \\
\end{eqnarray*}
Those frequencies probably look pretty familiar to you. They are, of
course, the familiar Hardy-Weinberg proportions. But we're not done
yet. In order to say that these proportions will also be the genotype
proportions of adults in the progeny generation, we have to make two
more assumptions:

\begin{description}

\item[Assumption \#7] Generations do not overlap.\footnote{Or the
    allele frequency is the same in generations that do overlap.}

\item[Assumption \#8] There are no differences among genotypes in the
probability of survival.

\end{description}

\section*{The Hardy-Weinberg principle}\index{Hardy-Weinberg principle}

After a single generation in which {\it all\/} eight of the above
assumptions are satisfied

\begin{eqnarray}
\hbox{freq.}(A_1A_1\hbox{ in adults}) &=& p^2 \label{eq:hw-p2} \\
\hbox{freq.}(A_1A_2\hbox{ in adults}) &=& 2pq \label{eq:hw-2pq} \\
\hbox{freq.}(A_2A_2\hbox{ in adults}) &=& q^2 \label{eq:hw-q2}
\end{eqnarray}

\noindent It's vital to understand the logic here.

\begin{enumerate}

\item If Assumptions \#1--\#8 are true, then equations
  \ref{eq:hw-p2}--\ref{eq:hw-q2} {\bf must} be true.

\item If genotypes are {\it not\/} in Hardy-Weinberg proportions, one
  or more of Assumptions \#1--\#8 {\bf must} be false.

\item If genotypes are in Hardy-Weinberg proportions, one or more of
  Assumptions \#1--\#8 may still be violated.

\item Assumptions \#1--\#8 are {\it sufficient\/} for Hardy-Weinberg
  to hold, but they are not {\it necessary\/} for Hardy-Weinberg to
  hold.

\end{enumerate}

Point (2) is why the Hardy-Weinberg principle is so important. There
isn't a population of any organism anywhere in the world that
satisfies all 8 assumptions, even for a single
generation.\footnote{There may be some that come reasonably close, but
  none that fulfill them {\it exactly}. There aren't any populations
  of infinite size, for example.}  But {\it all\/} possible
evolutionary processes within populations cause a violation of at
least one of these assumptions. Departures from Hardy-Weinberg are one
way in which we can detect those processes and estimate their
magnitude.\footnote{Actually, there's a ninth assumption that I didn't
  mention. Everything I said here depends on the assumption that the
  locus we're dealing with is autosomal. We can talk about what
  happens with sex-linked loci, if you want. But again, mostly what we
  get is algebraic complications without a lot of new insight.}

\section*{Estimating allele frequencies}

Before we can determine whether genotypes in a population are in
Hardy-Weinberg proportions, we need to be able to estimate the
frequency of both genotypes and alleles. This is easy when you can
identify all of the alleles within genotypes, but suppose that we're
trying to estimate allele frequencies in the ABO blood group system in
humans. Then we have a situation that looks like this:

\begin{center}
\begin{tabular}{l|r|r|r|r}
\hline\hline
Phenotype      & A      & AB       & B       & O  \\
\hline
Genotype(s)    & aa\ ao & ab       & bb\ bo  & oo \\
No.\ in sample & $N_A$  & $N_{AB}$ & $N_{B}$ & $N_O$ \\
\hline
\end{tabular}
\end{center}
Now we can't directly count the number of $a$, $b$, and $o$
alleles. What do we do? Well, more than 50 years ago, some geneticists
figured out how with a method they called ``gene
counting''~\cite{Ceppellini-etal-1955} and that statisticians later
generalized for a wide variety of purposes and called the EM
algorithm~\cite{Dempster-etal-1977}. It uses a trick you'll see
repeatedly through this course. When we don't know something we want
to know, we pretend that we know it and do some calculations with what
we just pretended to know. If we're lucky, we can fiddle with our
calculations a bit to relate the thing that we pretended to know to
something we actually do know so we can figure out what we wanted to
know. Make sense? Probably not. Let's try an example and see if that
helps.\index{EM algorithm}

If we knew $p_a$, $p_b$, and $p_o$, we could figure out how many
individuals with the $A$ phenotype have the $aa$ genotype and how many
have the $ao$ genotype, namely
\begin{eqnarray*}
N_{aa} &=& n_A \left({p_a^2 \over p_a^2 + 2p_ap_o}\right) \\
N_{ao} &=& n_A \left({2p_ap_o \over p_a^2 + 2p_ap_o}\right) \quad .
\end{eqnarray*}
Obviously we could do the same thing for the $B$ phenotype:
\begin{eqnarray*}
N_{bb} &=& n_B \left({p_b^2 \over p_b^2 + 2p_bp_o}\right) \\
N_{bo} &=& n_B \left({2p_bp_o \over p_b^2 + 2p_bp_o}\right) \quad .
\end{eqnarray*}
Notice that $N_{ab} = N_{AB}$ and $N_{oo} = N_O$~(lowercase
subscripts refer to genotypes, uppercase to phenotypes). If we knew
all this, then we could calculate $p_a$, $p_b$, and $p_o$ from
\begin{eqnarray*}
p_a &=& {2N_{aa} + N_{ao} + N_{ab} \over 2N} \\
p_b &=& {2N_{bb} + N_{bo} + N_{ab} \over 2N} \\
p_o &=& {2N_{oo} + N_{ao} + N_{bo} \over 2N} \quad ,
\end{eqnarray*}
where $N$ is the total sample size.

Surprisingly enough we can actually estimate the allele frequencies by
using this trick. Just take a guess at the allele frequencies. Any
guess will do. Then calculate $N_{aa}$, $N_{ao}$, $N_{bb}$, $N_{bo}$,
$N_{ab}$, and $N_{oo}$ as described in the preceding
paragraph.\footnote{Chances are $N_{aa}$, $N_{ao}$, $N_{bb}$, and
  $N_{bo}$ won't be integers. That's OK. Pretend that there really are
  fractional animals or plants in your sample and proceed.} That's the
{\bf E}xpectation part the EM algorithm. Now take the values for
$N_{aa}$, $N_{ao}$, $N_{bb}$, $N_{bo}$, $N_{ab}$, and $N_{oo}$ that
you've calculated and use them to calculate new values for the allele
frequencies. That's the {\bf M}aximization part of the EM
algorithm. It's called ``maximization'' because what you're doing is
calculating maximum-likelihood estimates of the allele frequencies,
given the observed (and made up) genotype counts.\footnote{If you
  don't know what maximum-likelihood estimates are, don't worry. We'll
  get to that in a moment.} Chances are your new values for $p_a$,
$p_b$, and $p_o$ won't match your initial guesses, but\footnote{Yes,
  truth {\it is\/} sometimes stranger than fiction.}  if you take
these new values and start the process over and repeat the whole
sequence several times, eventually the allele frequencies you get out
at the end match those you started with. These are maximum-likelihood
estimates of the allele frequencies.\footnote{I should point out that
  this method {\it assumes\/} that genotypes are found in
  Hardy-Weinberg proportions.}

Consider the following example:
\begin{center}
\begin{tabular}{l|rrrr}
\hline\hline
Phenotype      & A      & AB      & AB     & O  \\
No.\ in sample & 25     & 50      & 25     & 15 \\
\hline
\end{tabular}
\end{center}
We'll start with the guess that $p_a = 0.33$, $p_b = 0.33$, and
$p_o = 0.34$. With that assumption we would calculate that
$25(0.33^2/(0.33^2 + 2(0.33)(0.34))) = 8.168$ of the A phenotypes in
the sample have genotype $aa$, and the remaining 16.832 have genotype
$ao$. Similarly, we can calculate that 8.168 of the B phenotypes in
the population sample have genotype $bb$, and the remaining 16.832
have genotype $bo$. Now that we have a guess about how many
individuals of each genotype we have,\footnote{Since we're making
  these genotype counts up, we can also pretend that it makes sense to
  have fractional numbers of genotypes.} we can calculate a new guess
for the allele frequencies, namely $p_a = 0.362$, $p_b = 0.362$, and
$p_o = 0.277$. By the time we've repeated this process four more
times, the allele frequencies aren't changing anymore, and the maximum
likelihood estimate of the allele frequencies is $p_a = 0.372$,
$p_b = 0.372$, and $p_o = 0.256$.

\subsection*{What is a maximum-likelihood
  estimate?}\index{maximum-likelihood estimates}

I just told you that the method I described produces
``maximum-likelihood estimates'' for the allele frequencies, but I
haven't told you what a maximum-likelihood estimate {\it is\/}. The
good news is that you've been using maximum-likelihood estimates for
as long as you've been estimating anything, without even knowing
it. Although it will take me a while to explain it, the idea is
actually pretty simple.

Suppose we had a sock drawer with two colors of socks, red and
green. And suppose we were interested in estimating the proportion of
red socks in the drawer. One way of approaching the problem would be
to mix the socks well, close our eyes, take one sock from the drawer,
record its color and replace it. Suppose we do this $N$ times. We know
that the number of red socks we'll get might be different the next
time, so the number of red socks we actually get is a random
variable. Let's call that random variable $K$. Now suppose in our
actual experiment we find $k$ red socks, i.e., the value our random
variable takes on is $k$ or putting it in an equation: $K=k$. If we
knew $p$, the proportion of red socks in the drawer, we could
calculate the probability of getting the data we observed, namely
\begin{equation}
\mbox{P}(K=k|p) = {N \choose k} p^k (1-p)^{(N-k)} \quad . \label{eq:binomial}
\end{equation}
This is the {\it binomial probability distribution}. The part on the
left side of the equation is read as ``The probability that we get $k$
red socks in our sample {\it given\/} the value of $p$.'' The word
``given'' means that we're calculating the probability of our data
conditional on the (unknown) value $p$.

Of course we don't know $p$, so what good does
writing~(\ref{eq:binomial}) do? Well, suppose we reverse the question
to which equation~(\ref{eq:binomial}) is an answer and call the
expression in~(\ref{eq:binomial}) the ``likelihood of the data.''
Suppose further that we find the value of $p$ that makes the
likelihood bigger than any other value we could
pick.\footnote{Technically, we treat $\mbox{P}(K=k|p)$ as a function
  of $p$, find the value of $p$ that maximizes it, and call that value
  $\hat p$.} Then $\hat p$ is the maximum-likelihood estimate of
$p$.\footnote{You'll be relieved to know that in this case, $\hat p =
  k/N$.}

In the case of the ABO blood group that we just talked about, the
likelihood is a bit more complicated
\begin{equation}
{N \choose N_A N_{AB} N_B N_O}
\left(p_a^2 + 2p_ap_o\right)^{N_A}
2p_ap_b^{N_{AB}}
\left(p_b^2 + 2p_bp_o\right)^{N_B}
\left(p_o^2\right)^{N_O}
\end{equation}
This is a {\it multinomial probability distribution}. It turns out
that one way to find the values of $p_a$, $p_b$, and $p_o$ is to use
the EM algorithm I just described.\footnote{There's another way I'd be
  happy to describe if you're interested, but it's a lot more
  complicated.} There isn't a simple formula that allows us to write
down an expression for the maximum-likelihood estimate of the allele
frequencies in terms of the phenotype frequencies. We have to use an
algorithm to find them, and the EM algorithm happens to be a
particularly convenient algorithm to use. 

\section*{An introduction to Bayesian inference}\index{Bayesian inference}

Maximum-likelihood estimates have a lot of nice features, but they are
also a slightly backwards way of looking at the world. The likelihood
of the data is the probability of the data, $x$, given parameters that
we don't know, $\phi$, i.e, $\mbox{P}(x|\phi)$. It seems a lot more
natural to think about the probability that the unknown parameter
takes on some value, given the data, i.e.,
$\mbox{P}(\phi|x)$. Surprisingly, these two quantities are closely
related. Bayes' Theorem tells us that
\begin{equation}
\mbox{P}(\phi|x) = \frac{\mbox{P}(x|\phi)\mbox{P}(\phi)}{\mbox{P}(x)} \quad .
\label{eq:bayes}
\end{equation}
We refer to $\mbox{P}(\phi|x)$ as the {\it posterior distribution} of
$\phi$, i.e., the probability that $\phi$ takes on a particular value
given the data we've observed, and to $\mbox{P}(\phi)$ as the {\it
  prior distribution} of $\phi$, i.e., the probability that $\phi$
takes on a particular value {\it before\/} we've looked at any
data. Notice how the relationship in~(\ref{eq:bayes}) mimics the logic
we use to learn about the world in everyday life. We start with some
prior beliefs, $\mbox{P}(\phi)$, and modify them on the basis of data
or experience, $\mbox{P}(x|\phi)$, to reach a conclusion,
$\mbox{P}(\phi|x)$. That's the underlying logic of Bayesian
inference.\index{Bayesian inference}

\subsection*{Estimating allele frequencies with two alleles}

Let's suppose we've collected data from a population of {\it Protea
  repens}\footnote{A few of you may recognize that I didn't choose
  that species entirely at random, even though the ``data'' I'm
  presenting here are entirely fanciful.} and have found 7 alleles
coding for the {\it fast\/} allele at a enzyme locus encoding
glucose-phosphate isomerase in a sample of 20 alleles. We want to
estimate the frequency of the {\it fast\/} allele. The
maximum-likelihood estimate is $7/20 = 0.35$, which we got by finding
the value of $p$ that maximizes
\begin{eqnarray*}
\mbox{P}(k|N,p) &=& {N \choose k} p^k (1-p)^{N-k} \quad ,
\end{eqnarray*}
where $N=20$ and $k=7$. A Bayesian uses the same likelihood, but has
to specify a prior distribution for $p$. If we didn't know anything
about the allele frequency at this locus in {\it P. repens} before
starting the study, it makes sense to express that ignorance by
choosing $\mbox{P}(p)$ to be a uniform random variable on the interval
$[0,1]$. That means we regarded all values of $p$ as equally likely
prior to collecting the data.\footnote{If we had prior information
  about the likely values of $p$, we'd pick a different prior
  distribution to reflect our prior information. See the Summer
  Institute notes for more information, if you're interested.}

Until the early 1990s\footnote{You are probably thinking to yourself
  ``The 1990s? That's ancient history. Why is Holsinger making such a
  big deal about this'' Please cut me a little slack. I know that most
  of you weren't born in the early 90s, but I'd already taught this
  course two or three times by the time the paper I'm about to refer
  to was published.} it was necessary to do a bunch of complicated
calculus to combine the prior with the likelihood to get a
posterior. Since the early 1990s statisticians have used a simulation
approach, Monte Carlo Markov Chain sampling, to construct numerical
samples from the posterior. For the problems encountered in this
course, we'll mostly be using the freely available software package
{\tt Stan} through its interface in {\tt R}, {\tt rstan}, to implement
Bayesian analyses. For the problem we just encountered, here's the
code that's needed to get our results:\footnote{This code and other
  {\tt Stan} code used in the course can be found on the course web
  site by following the links associated with the corresponding
  lecture.}\index{Stan@\texttt{Stan}}\index{MCMC sampling}
\begin{verbatim}
data {
  int<lower=0> N;     // the sample size
  int<lower=0> k;     // the number of A_1 alleles observed
}

parameters {
  real<lower=0, upper=1> p;  // the allele frequency
}

model {
  // likelihood
  //
  k ~ binomial(N, p);

  // prior
  p ~ uniform(0.0, 1.0);
}
\end{verbatim}
We can run this is in {\tt R} by {\tt source()}'ing the following
code. Remember that in our fictitious example, we found 7 fast alleles
in a sample of 20, i.e., $k=7$ and $N=20$.
\begin{verbatim}
## Load the rstan library
##
library(rstan)

## set the number of chains to the number of cores in the computer
##
options(mc.cores = parallel::detectCores())

## set up the data
##   N: sample size
##   k: number of A1 alleles
stan_data <- list(N = 20,
                  k = 7)

## Invoke stan
##
fit <- stan("binomial-model.stan",
            data = stan_data,
            refresh = 0)

## print the results on the console with 3 digits after the decimal
##
print(fit, digits = 3)
\end{verbatim}

\noindent Here's what you'll see in the terminal.\footnote{Your
  computer may appear to freeze after the message about avoiding
  recompilation. Don't worry. It's just thinking.}

\begin{verbatim}
> source("binomial-model.R")
Loading required package: StanHeaders
Loading required package: ggplot2
rstan (Version 2.21.2, GitRev: 2e1f913d3ca3)
For execution on a local, multicore CPU with excess RAM we recommend calling
options(mc.cores = parallel::detectCores()).
To avoid recompilation of unchanged Stan programs, we recommend calling
rstan_options(auto_write = TRUE)
Inference for Stan model: binomial-model.
4 chains, each with iter=2000; warmup=1000; thin=1; 
post-warmup draws per chain=1000, total post-warmup draws=4000.

        mean se_mean    sd    2.5%     25%     50%     75%   97.5% n_eff  Rhat
p      0.360   0.003 0.099   0.179   0.289   0.357   0.424   0.561  1475 1.001
lp__ -14.926   0.017 0.719 -16.901 -15.088 -14.646 -14.470 -14.421  1691 1.000

Samples were drawn using NUTS(diag_e) at Sat Jun  5 16:54:55 2021.
For each parameter, n_eff is a crude measure of effective sample size,
and Rhat is the potential scale reduction factor on split chains (at 
convergence, Rhat=1).
>
\end{verbatim}

Most of the column headings should be fairly self-explanatory. {\tt
  mean} is our best guess for the value for the frequency of the {\it
  fast\/} allele, the posterior mean of $p$. {\tt sd} is the posterior
standard deviation of $p$. It's our best guess of the uncertainty
associated with our estimate of the frequency of the {\it fast\/}
allele. The 2.5\%, 50\%, and 97.5\% columns are the percentiles of the
posterior distribution. The [2.5\%, 97.5\%] interval is the 95\%
credible interval, which is analogous to the 95\% confidence interval
in classical statistics, except that we can say that there's a 95\%
chance that the frequency of the {\it fast\/} allele lies within this
interval.\footnote{If you don't understand why that's different from a
  standard confidence interval, ask me about it.} Since the results
are from a simulation, different runs will produce slightly different
results. In this case, we have a posterior mean of about 0.36 (as
opposed to the maximum-likelihood estimate of 0.35), and there is a
95\% chance that $p$ lies in the interval [0.18, 0.56].

\section*{Returning to the ABO example}

Here's data from the ABO blood group:\footnote{This is almost the last
time! I promise.}
\begin{center}
\begin{tabular}{l|ccccc}
\hline\hline
Phenotype &   A &  AB &   B &   O & Total \\
Observed  & 862 & 131 & 365 & 702 & 2060 \\
\hline
\end{tabular}
\end{center}
To estimate the underlying allele frequencies, $p_A$, $p_B$, and
$p_O$, we have to remember how the allele frequencies map to phenotype
frequencies:\footnote{Assuming genotypes are in Hardy-Weinberg
  proportions. We'll relax that assumption later.}
\begin{eqnarray*}
\hbox{Freq}(A) &=& p_A^2 + 2p_Ap_O \\
\hbox{Freq}(AB) &=& 2p_Ap_B \\
\hbox{Freq}(B) &=& p_B^2 + 2p_Bp_O \\
\hbox{Freq}(O) &=& p_O^2 \quad .
\end{eqnarray*}
Hers's the {\tt Stan} code we use to estimate the allele
frequencies:
\begin{verbatim}
data {
  int<lower=0> N_A;
  int<lower=0> N_AB;
  int<lower=0> N_B;
  int<lower=0> N_O;
}

transformed data {
  int<lower=0> N[4];

  N[1] = N_A;
  N[2] = N_AB;
  N[3] = N_B;
  N[4] = N_O;
}

parameters {
  // the three allele frequencies add to 1
  //
  simplex[3] p;
}

transformed parameters {
  real<lower=0, upper=1> p_a;
  real<lower=0, upper=1> p_b;
  real<lower=0, upper=1> p_o;
  // the four phenotype frequencies add to 1
  //
  simplex[4] x;

  // allele frequencies
  //
  p_a = p[1];
  p_b = p[2];
  p_o = p[3];
  // phenotype frequencies
  //
  // A
  x[1] = p_a^2 + 2*p_a*p_o;
  // AB
  x[2] = 2*p_a*p_b;
  // B
  x[3] = p_b^2 + 2*p_b*p_o;
  // O
  x[4] = p_o^2;
}

model {
  // likelihood
  //
  N ~ multinomial(x);

  // prior
  //
  p ~ dirichlet(rep_vector(1.0, 3));
}
\end{verbatim}
The {\tt dirichlet()} prior produces a uniform distribution across all
three allele frequencies while ensuring that they sum to 1. Here are
the results of the analysis:
{\footnotesize
\begin{verbatim}
> source("abo-model.R")
Inference for Stan model: abo-model.
4 chains, each with iter=2000; warmup=1000; thin=1; 
post-warmup draws per chain=1000, total post-warmup draws=4000.

          mean se_mean    sd      2.5%       25%       50%       75%     97.5% n_eff  Rhat
p[1]     0.281   0.000 0.008     0.266     0.276     0.281     0.287     0.297  3814 1.000
p[2]     0.129   0.000 0.005     0.119     0.126     0.129     0.133     0.140  3685 1.000
p[3]     0.589   0.000 0.008     0.573     0.584     0.589     0.595     0.605  3428 1.001
p_a      0.281   0.000 0.008     0.266     0.276     0.281     0.287     0.297  3814 1.000
p_b      0.129   0.000 0.005     0.119     0.126     0.129     0.133     0.140  3685 1.000
p_o      0.589   0.000 0.008     0.573     0.584     0.589     0.595     0.605  3428 1.001
x[1]     0.411   0.000 0.010     0.391     0.404     0.411     0.418     0.431  4033 1.000
x[2]     0.073   0.000 0.003     0.067     0.071     0.073     0.075     0.079  3417 1.001
x[3]     0.169   0.000 0.007     0.156     0.164     0.169     0.174     0.183  3764 1.000
x[4]     0.347   0.000 0.010     0.328     0.341     0.347     0.354     0.366  3430 1.001
lp__ -2506.009   0.022 0.963 -2508.531 -2506.366 -2505.715 -2505.316 -2505.068  1915 1.000

Samples were drawn using NUTS(diag_e) at Sat Jun  5 17:23:22 2021.
For each parameter, n_eff is a crude measure of effective sample size,
and Rhat is the potential scale reduction factor on split chains (at 
convergence, Rhat=1).
>
\end{verbatim}
}
\noindent The posterior means for the allele frequencies are indistinguishable
from the maximum-likelihood estimates ($p_a = 0.281$, $p_b = 0.129$,
and $p_o = 0.59$), but we also have 95\% credible intervals so that we
have an assessment of how reliable the Bayesian estimates are. We also
have estimates of the phenotype frequencies and their
reliability. Getting estimates of the reliability for the allele
frequencies from a likelihood analysis is possible, but it takes a
fair amount of additional work.

\bibliography{popgen}
\bibliographystyle{plain}

\ccLicense

\end{document}
