\documentclass[12pt]{article}
\usepackage{lecture}
\usepackage{html}
\usepackage{graphics}

\newcommand{\copyrightYears}{2001-2021}
\newcommand{\htmladdlinktext}[1]{\htmladdnormallink{#1}{#1}}

\title{Inbreeding and self-fertilization}

\begin{document}

\maketitle

\thispagestyle{first}

\section*{Introduction}

Remember that long list of assumptions associated with derivation of
the Hardy-Weinberg principle that we just finished? Well, we're about
to begin violating assumptions to explore the consequences, but we're
not going to violate them in order. We're first going to violate
Assumption \#2:\index{inbreeding}

\begin{quote}
Genotypes mate at random with respect to their genotype at this
particular locus.
\end{quote}

\noindent There are many ways in which this assumption might be
violated:

\begin{itemize}

\item Some genotypes may be more successful in mating than
  others{\dash}sexual selection.\index{sexual selection}

\item Genotypes that are different from one another may mate more
  often than expected{\dash}disassortative mating, e.g.,
  self-incompatibility alleles in flowering plants, MHC loci in
  humans~(the smelly t-shirt
  experiment~\cite{Wedekind-etal-1995}).\index{assortative mating}

\item Genotypes that are similar to one another may mate more often
  than expected{\dash}assortative mating.

\item Some fraction of the offspring produced may be produced
  asexually.

\item Individuals may mate with
  relatives{\dash}inbreeding.\index{inbreeding!types}

\begin{itemize}

\item self-fertilization

\item sib-mating

\item first-cousin mating

\item parent-offspring mating

\item etc.

\end{itemize}

\end{itemize}

When there is sexual selection or disassortative mating genotypes
differ in their chances of being included in the breeding
population. As a result, allele and genotype frequencies will tend to
change from one generation to the next. We'll talk a little about
these types of departures from random mating when we discuss the
genetics of natural selection in a few weeks, but we'll ignore them
for now. In fact, we'll also ignore assortative mating, since it's
properties are fairly similar to those of inbreeding, and inbreeding
is easier to understand. We'll also ignore asexual reproduction, since
genotypes simply reproduce themselves and the genetic composition of
the population doesn't change.\footnote{Assuming, of course, that all
  of the other assumptions underlying Hardy-Weinberg continue to
  apply. In the real world, the genetic composition of the population
  will change, but we're not going to discuss how asexual reproduction
  influences changes in the genotype composition of populations unless
  there is overwhelming demand to do so.}

\section*{Self-fertilization}

Self-fertilization is the most extreme form of inbreeding
possible,\index{inbreeding!self-fertilization}\index{self-fertilization}
and it is characteristic of many flowering plants and some
hermaphroditic animals, including freshwater snails and that darling
of developmental genetics, {\it Caenorhabditis elegans}.\footnote{It
  could be that it is characteristic of {\it many\/} hermaphroditic
  animal parasites, but I'm a plant biologist. I know next to nothing
  about animal mating systems, so I don't have a good feel for how
  extensively self-fertilization has been looked for in hermaphroditic
  animals. You should also know that I exaggerated when I wrote that
  ``self-fertilization is the most extreme form of inbreeding.''
  (Watch me carefully. I have a tendency to exaggerate in the main
  text of these notes. I usually try to provide the complicating
  details in footnotes in the hope that they'll be less distracting
  here.) The form of self-fertilization I'm going to describe actually
  isn't the most extreme form of self-fertilization possible. That
  honor belongs to gametophytic self-fertilization in homosporous
  plants. The offspring of gametophytic self-fertilization are
  uniformly homozygous at every locus in the genome. If you don't know
  what gametophytic self-fertilization is, you're not alone. Ask me or
  see~\cite{Holsinger-1990} if your want more details.} It's not too
hard to figure out what the consequences of self-fertilization will be
without doing any algebra.\footnote{As you'll see, though, I often
  resort to algebra, because it makes things even clearer.}

\begin{itemize}

\item All progeny of homozygotes are themselves homozygous.

\item Half of the progeny of heterozygotes are heterozygous and half
are homozygous.

\end{itemize}
So you'd expect that the frequency of heterozygotes would be halved
every generation, that the frequency of homozygotes would increase,
and that the allele frequencies wouldn't change,\footnote{Since half
  of the homozygous offspring carry one of the two alleles and the
  other half carry the other one, the overall frequency of alleles
  doesn't change.} and you'd be right. To see why, consider the
following mating table:\footnote{Note: The ``missing'' entries in the
  mating table are mating events that never happen.}\index{mating table!self-fertilization}

\begin{center}
\begin{tabular}{ccccc}
\hline\hline
&&\multicolumn{3}{c}{Offsrping genotype} \\
Mating & frequency & $A_1A_1$ & $A_1A_2$ & $A_2A_2$ \\
\hline
$A_1A_1 \times A_1A_1$ & $x_{11}$ & 1 & 0 & 0 \\
$A_1A_2 \times A_1A_2$ & $x_{12}$ & $\fourth$ & $\half$ & $\fourth$ \\
$A_2A_2 \times A_2A_2$ & $x_{22}$ & 0 & 0 & 1 \\
\hline
\end{tabular}
\end{center}

\noindent Using the same technique we used to derive the
Hardy-Weinberg principle, we can calculate the frequency of the
different offspring genotypes from the above table.
\begin{eqnarray}
x_{11}' &=& x_{11} + x_{12}/4 \\
x_{12}' &=& x_{12}/2 \\
x_{22}' &=& x_{22} + x_{12}/4
\end{eqnarray}

\noindent I use the $'$ to indicate the next generation. Notice that
in making this caclulation I assume that all other conditions
associated with Hardy-Weinberg apply (meiosis is fair, no differences
among genotypes in probability of survival, no input of new genetic
material, etc.). We can also calculate the frequency of the $A_1$
allele among offspring, namely
\begin{eqnarray}
p' &=& x_{11}' + x_{12}'/2 \\
   &=& x_{11} + x_{12}/4 + x_{12} /4 \\
   &=& x_{11} + x_{12}/2 \\
   &=& p
\end{eqnarray}

These equations illustrate two very important principles that are true
with any system of strict inbreeding:\index{inbreeding!consequences}

\begin{enumerate}

\item Inbreeding does not cause {\it allele\/} frequencies to change,
  but it will generally cause {\it genotype\/} frequencies to
  change.\footnote{This is why I think of evolution as a change in the
    genotypic composition of a population over time, {\it not} as a
    change in allele frequencies over time. I think a population that
    is self-fertilizing evolves, because the genotype frequencies
    change even though the allele frequencies don't.}

\item Inbreeding reduces the frequency of heterozygotes relative to
  Hardy-Weinberg expectations. It need not eliminate heterozygotes
  entirely, but it is guaranteed to reduce their
  frequency.\footnote{Inbreeding that leads to distinct family lines,
    e.g., self-fertilization or sib-mating, will completely eliminate
    heterozygosity over time.}

\begin{itemize}

\item Suppose we have a population of hermaphrodites in which $x_{12}
= 0.5$ and we subject it to strict self-fertilization. Assuming that
inbred progeny are as likely to survive and reproduce as outbred
progeny, $x_{12} < 0.01$ in six generations and $x_{12} < 0.0005$ in
ten generations.

\end{itemize}

\end{enumerate}

\section*{Partial self-fertilization}

Many plants reproduce by a mixture of outcrossing and
self-fertilization.\index{inbreeding!partial
  self-fertilization}\index{self-fertilization!partial} To a
population geneticist that means that they reproduce by a mixture of
selfing and random mating.\footnote{It would be more accurate to
  write: ``Population geneticists usually model partial
  self-fertilization as a mixture of self-fertilization and random
  mating. That simple model ignores a lot of complexity in how
  self-fertilization happens, but it's a useful approximation for most
  purposes.''} Now I'm going to pull a fast one and derive the
equations that determine how allele frequencies change from one
generation to the next without using a mating table. To do so, I'm
going to imagine that our population consists of a mixture of two
populations. In one part of the population all of the reproduction
occurs through self-fertilization and in the other part all of the
reproduction occurs through random mating. If you think about it for a
while, you'll realize that this is equivalent to imagining that each
plant reproduces some fraction of the time through self-fertilization
and some fraction of the time through random mating.\footnote{Again,
  it would be more accurate to write: ``If you think about it for a
  while, you'll realize that for purposes of understanding how
  genotype frequencies change through time this is equivalent to
  assuming that each plant produces some fraction of its progeny
  through self-fertilization and some fraction through outcrossing.''}
Let $\sigma$ be the fraction of progeny produced through
self-fertilization, then
\begin{eqnarray}
x_{11}' &=& p^2(1-\sigma) + (x_{11} + x_{12}/4)\sigma \\
x_{12}' &=& 2pq(1-\sigma) + (x_{12}/2)\sigma  \label{eq:het} \\
x_{22}' &=& q^2(1-\sigma) + (x_{22} + x_{12}/4)\sigma
\end{eqnarray}
Notice that I use $p^2$, $2pq$, and $q^2$ for the genotype frequencies
in the part of the population that's mating at random. {\bf Question}:
Why can I get away with that?\footnote{If you're being good little
  boys and girls and looking over these notes {\it before\/} you get
  to class, when you see {\bf Question} in the notes, you'll know to
  think about that a bit, because I'm not going to give you the answer
  in the notes, I'm going to help you discover it during lecture.}

It takes a little more algebra than it did before, but it's not
difficult to verify that the allele frequencies don't change between
parents and offspring.
\begin{eqnarray}
p' &=& \left\{p^2(1-\sigma) + (x_{11} + x_{12}/4)\sigma\right\}
       + \left\{pq(1-\sigma) + (x_{12}/4)\sigma\right\} \\
   &=& p(p+q)(1-\sigma) + (x_{11} + x_{12}/2)\sigma \\
   &=& p(1-\sigma) + p\sigma \\
   &=& p
\end{eqnarray}
Because homozygous parents can always have heterozygous offspring
(when they outcross), heterozygotes are never completely eliminated
from the population as they are with complete self-fertilization. In
fact, we can solve for the {\it equilibrium} frequency of
heterozygotes, i.e., the frequency of heterozygotes reached when
genotype frequencies stop changing.\footnote{This is analogous to
  stopping the calculation and re-calculation of allele frequencies in
  the EM algorithm when the allele frequency estimates stop changing.}
By definition, an equilibrium for $x_{12}$ is a value such that if we
put it in on the right side of equation (\ref{eq:het}) we get it back
on the left side, or in equations\index{eqilibrium}
\begin{eqnarray}
\hat x_{12} &=& 2pq(1-\sigma) + (\hat x_{12}/2)\sigma \\
\hat x_{12}(1 - \sigma/2) &=& 2pq(1-\sigma) \\
\hat x_{12} &=& \frac{2pq(1-\sigma)}{(1-\sigma/2)}
\end{eqnarray}

\noindent It's worth noting several things about this set of
equations:

\begin{enumerate}

\item I'm using $\hat x_{12}$ to refer to the equilibrium frequency of
  heterozygotes. I'll be using hats over variables to denote
  equilibrium properties throughout the
  course.\footnote{Unfortunately, I'll also be using hats to denote
    estimates of unknown parameters, as I did when discussing
    maximum-likelihood estimates of allele frequencies. I apologize
    for using the same notation to mean different things, but I'm
    afraid you'll have to get used to figuring out the meaning from
    the context. Believe me. Things are about to get a lot worse. Wait
    until I tell you how many different ways population geneticists
    use a parameter $f$ that is commonly called the inbreeding
    coefficient.}

\item I can solve for $\hat x_{12}$ in terms of $p$ because I know
  that $p$ doesn't change. If $p$ changed, the calculations wouldn't
  be nearly this simple.

\item The equilibrium is approached gradually (or asymptotically as
  mathematicians would say). A single generation of random mating will
  put genotypes in Hardy-Weinberg proportions (assuming all the other
  conditions are satisfied), but many generations may be required for
  genotypes to approach their equilibrium frequency with partial
  self-fertilization.

\end{enumerate}

\section*{Inbreeding coefficients}\index{inbreeding coefficient}

Now that we've found an expression for $\hat x_{12}$ we can also find
expressions for $\hat x_{11}$ and $\hat x_{22}$. The complete set of
equations for the genotype frequencies with partial selfing are:
\begin{eqnarray}
\hat x_{11} &=& p^2 + \frac{\sigma pq}{2(1-\sigma/2)} \\
\hat x_{12} &=& 2pq - 2\left(\frac{\sigma pq}{2(1-\sigma/2)}\right) \\
\hat x_{22} &=& q^2 + \frac{\sigma pq}{2(1-\sigma/2)} 
\end{eqnarray}
Notice that all of those equations have a term
$\sigma/(2(1-\sigma/2))$. Let's call that term $f$. Then we can save
ourselves a little hassle by rewriting the above equations as:
\begin{eqnarray}
\hat x_{11} &=& p^2 + fpq \\
\hat x_{12} &=& 2pq(1-f) \\
\hat x_{22} &=& q^2 + fpq
\end{eqnarray}
Now you're going to have to stare at this a little longer, but notice
that $\hat x_{12}$ is the frequency of heterozygotes that we observe
and $2pq$ is the frequency of heterozygotes we'd expect under
Hardy-Weinberg in this population.\footnote{In both cases, I'm
  assuming that we have observed the genotype and allele frequencies
  without error. When we talk about estimating $f$ a little later,
  you'll see how things work in the real world~(as opposed to how they
  work in the imaginary world I'm fond of spending my time in).} So

\begin{eqnarray}
1-f &=& \frac{\hat x_{12}}{2pq} \\
  f &=& 1 - \frac{\hat x_{12}}{2pq} \\
    &=& 1 - \frac{\hbox{observed heterozygosity}}%
                 {\hbox{expected heterozygosity}}
\end{eqnarray}

$f$ is the inbreeding coefficient. When defined as 1 - (observed
heterozygosity)/(expected heterozygosity) it can be used to measure
the extent to which a particular population departs from
Hardy-Weinberg expectations.\footnote{$f$ can be negative if there are
  more heterozygotes than expected, as might be the case if
  cross-homozygote matings are more frequent than expected at random.}
When $f$ is defined in this way, I refer to it as the {\it population
  inbreeding coefficient}.\index{inbreeding
  coefficient!population}\footnote{To be honest, I'll {\it try\/} to
  remember to refer to it this way. Chances are that I'll forget
  sometimes and just call it the inbreeding coefficient. If I do,
  you'll either have to figure out what I mean from the context or ask
  me to be more explicit.}

But $f$ can also be regarded as a function of a particular system of
mating. With partial self-fertilization the population inbreeding
coefficient when the population has reached equilibrium is
$\sigma/(2(1-\sigma/2))$. When regarded as the inbreeding coefficient
predicted by a particular system of mating, I refer to it as the {\it
  equilibrium inbreeding coefficient}.\index{inbreeding coefficient!equilibrium}

We'll encounter at least two more definitions for $f$ once I've
introduced idea of identity by descent.

\section*{Identity by descent}

Self-fertilization is, of course, only one example of the general
phenomenon of inbreeding{\dash}non-random mating in which individuals
mate with close relatives more often than expected at random. We've
already seen that the consequences of inbreeding can be described in
terms of the inbreeding coefficient, $f$ and I've introduced you to
two ways in which $f$ can be defined.\footnote{See paragraphs above
describing the population and equilibrium inbreeding coefficient.} I'm
about to introduce you to one more, but first I have to tell you about
identity by descent.

\begin{quote}
  Two alleles at a single locus are {\it identical by descent\/} if
  they are identical copies of the same allele in some earlier
  generation, i.e., both are copies that arose by DNA replication from
  the same ancestral sequence without any intervening
  mutation.\index{identity by descent}
\end{quote}

We're more used to classifying alleles by {\it type\/} than by {\it
  descent}. Although we don't usually say it explicitly, we regard
two alleles as the ``same,'' i.e., identical by type, \index{ideniity
  by type} if they have the same phenotypic effects. Whether or not
two alleles are identical by descent, however, is a property of their
genealogical history, not of their phenotypic effects. Consider the
following two scenarios:

\begin{center}
\begin{tabular}{rcccc}
Identity by descent \\
      &            & $A_1$ & $\rightarrow$ & $A_1$ \\
      & $\nearrow$ &       &               & \\
$A_1$ &            &       &               & \\
      & $\searrow$ &       &               & \\
      &            & $A_1$ & $\rightarrow$ & $A_1$ 
\end{tabular}
\end{center}

\begin{center}
\begin{tabular}{rcccc}
Identity by type \\
      &            & $A_1$ & $\rightarrow$ & $A_1$ \\
      & $\nearrow$ &       &               & \\
$A_1$ &            &       &               & \\
      & $\searrow$ &       &               & \\
      &            & $A_2$ & $\rightarrow$ & $A_1$ \\
      & $\uparrow$ &       & $\uparrow$    & \\
      & mutation   &       & mutation      & 
\end{tabular}
\end{center}

In both scenarios, the alleles at the end of the process are identical
in type, i.e., they're both $A_1$ alleles and they have the same
phenotypic effect. In the second scenario, however, they are identical
in type only because one of the alleles has two mutations in its
history.\footnote{Notice that we could also have had each allele
  mutate independently to $A_2$.} So alleles that are identical by
descent will also be identical by type, but alleles that are identical
by type need not be identical by descent.\footnote{Systematists in the
  audience will recognize this as the problem of homoplasy.}

A third definition for $f$ is the probability that two alleles {\it
chosen at random\/} are identical by descent.\footnote{Notice that if
we adopt this definition for $f$ it can only take on values between 0
and 1. When used in the sense of a population or equilibrium
inbreeding coefficient, however, $f$ can be negative.}  Of course,
there are several aspects to this definition that need to be spelled
out more explicitly.\footnote{OK, maybe ``of course'' is overstating
  it. It isn't really obvious that more clarity is needed until I
  point out the ambiguities in the bullet points that follow.}

\begin{itemize}

\item In what sense are the alleles chosen at random, within an
individual, within a particular population, within a particular set of
populations? 

\item How far back do we trace the ancestry of alleles to determine
whether they're identical by descent? Two alleles that are identical
by type may not share a common ancestor if we trace their ancestry
only 20 generations, but they may share a common ancestor if we trace
their ancestry back 1000 generations and neither may have undergone
any mutations since they diverged from one another.

\end{itemize}

Let's imagine for a moment, however, that we've traced back the
ancestry of all alleles in a particular population to what we call a
{\it reference population\/}, i.e., a population in which we regard
all alleles as unrelated.\index{reference population}\index{inbreeding!reference population} That's equivalent
to saying that alleles chosen at random from this population have zero
probability of being identical by descent, even if they are identical
by type. Given this assumption we
can write down the genotype frequencies in a descendant population
once we know $f$, where we define $f$ as the probability that two
alleles chosen at random in the descendant population are identical by
descent, i.e., descended from just one of the alleles in the reference
population.
\begin{eqnarray}
x_{11} &=& p^2(1-f) + fp \\
x_{12} &=& 2pq(1-f) \\
x_{22} &=& q^2(1-f) + fq \quad .
\end{eqnarray}
It may not be immediately apparent, but you've actually seen these
equations before in a different form.  Since $p - p^2 = p(1-p) = pq$
and $q - q^2 = q(1-q) = pq$ these equations can be rewritten as
\begin{eqnarray}
x_{11} &=& p^2 + fpq \\
x_{12} &=& 2pq(1-f) \\
x_{22} &=& q^2 + fpq \quad .
\end{eqnarray}

Now you can probably see why population geneticists tend to play fast
and loose with the definitions. {\it If\/} we ignore the distinction
between identity by type and identity by descent, then the equations
we used earlier to show the relationship between genotype frequencies,
allele frequencies, and $f$ (defined as a measure of departure from
Hardy-Weinberg expectations) are identical to those used to show the
relationship between genotype frequencies, allele frequencies, and $f$
(defined as a the probability that two randomly chosen alleles in the
population are identical by descent).

\bibliography{popgen}
\bibliographystyle{plain}

\ccLicense
                
\end{document}
