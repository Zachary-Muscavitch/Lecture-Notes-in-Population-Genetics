\documentclass[12pt]{article}
\usepackage{lecture}
\usepackage{graphics}
\usepackage{html}
\usepackage{url}

\newcommand{\copyrightYears}{2001-2021}

\title{Patterns of nucleotide and amino acid substitution}

\begin{document}

\maketitle

\thispagestyle{first}

\section*{Introduction}

So I've just suggested that the neutral theory of molecular evolution
explains quite a bit, but it also ignores quite a bit. The derivations
we did assumed that all substitutions are equally likely to occur,
because they are selectively neutral. That isn't plausible. We need
look no further than sickle cell anemia to see an example of a protein
polymorphism in which a single nucleotide substitution and a single
amino acid difference has a very large effect on fitness. Even
reasoning from first principles we can see that it doesn't make much
sense to think that all nucleotide substitutions are created
equal. Just as it's unlikely that you'll improve the performance of
your car if you pick up a sledgehammer, open its hood, close your
eyes, and hit something inside, so it's unlikely that picking a random
amino acid in a protein and substituting it with a different one will
improve the function of the protein.\footnote{Obviously it happens
  sometimes. If it didn't, there wouldn't be any adaptive
  evolution. It's just that, on average, mutations are more likely to
  decrease fitness than to increase it. }\index{sledgehammer
  principle}

\section*{The genetic code}

Of course, not all nucleotide sequence substitutions lead to amino
acid substitutions in protein-coding genes. There is redundancy in the
genetic code. Table~\ref{table:code} is a list of the codons in the
universal genetic code.\footnote{By the way, the ``universal'' genetic
  code is not universal. There are at least
  31~(\url{https://www.ncbi.nlm.nih.gov/Taxonomy/Utils/wprintgc.cgi}),
  but all of them have similar redundancy properties.} Notice that
there are only two amino acids, methionine and tryptophan, that have a
single codon. All the rest have at least two. Serine, arginine, and
leucine have six.\index{genetic code}

\begin{table}
\begin{center}
\begin{tabular}{llllllll}
\hline\hline
      & Amino &       & Amino &       & Amino &       & Amino \\
Codon & Acid  & Codon & Acid  & Codon & Acid  & Codon & Acid \\
\hline
UUU   & Phe   & UCU   & Ser   & UAU   & Tyr   & UGU   & Cys \\
UUC   & Phe   & UCC   & Ser   & UAC   & Tyr   & UGC   & Cys \\
UUA   & Leu   & UCA   & Ser   & UAA   & Stop  & UGA   & Stop \\
UUG   & Leu   & UCG   & Ser   & UAG   & Stop  & UGG   & Trp \\
      &       &       &       &       &       &       & \\
CUU   & Leu   & CCU   & Pro   & CAU   & His   & CGU   & Arg \\
CUC   & Leu   & CCC   & Pro   & CAC   & His   & CGC   & Arg \\
CUA   & Leu   & CCA   & Pro   & CAA   & Gln   & CGA   & Arg \\
CUG   & Leu   & CCG   & Pro   & CAG   & Gln   & CGG   & Arg \\
      &       &       &       &       &       &       & \\
AUU   & Ile   & ACU   & Thr   & AAU   & Asn   & AGU   & Ser \\
AUC   & Ile   & ACC   & Thr   & AAC   & Asn   & AGC   & Ser \\
AUA   & Ile   & ACA   & Thr   & AAA   & Lys   & AGA   & Arg \\
AUG   & Met   & ACG   & Thr   & AAG   & Lys   & AGG   & Arg \\
      &       &       &       &       &       &       & \\
GUU   & Val   & GCU   & Ala   & GAU   & Asp   & GGU   & Gly \\
GUC   & Val   & GCC   & Ala   & GAC   & Asp   & GGC   & Gly \\
GUA   & Val   & GCA   & Ala   & GAA   & Glu   & GGA   & Gly \\
GUG   & Val   & GCG   & Ala   & GAG   & Glu   & GGG   & Gly \\
\hline
\end{tabular}
\end{center}
\caption{The universal genetic code.}\label{table:code}
\end{table}

Moreover, most of the redundancy is in the third position, where we
can distinguish 2-fold from 4-fold redundant
sites~(Table~\ref{table:fold}). 2-fold redundant sites are those at
which either one of two nucleotides can be present in a codon for a
single amino acid. 4-fold redundant sites are those at which any of
the four nucleotides can be present in a codon for a single amino
acid. In some cases there is redundancy in the first codon position,
e.g, both AGA and CGA are codons for arginine. Thus, many nucleotide
substitutions at third positions do not lead to amino acid
substitutions, and some nucleotide substitutions at first positions do
not lead to amino acid substitutions. But every nucleotide
substitution at a second codon position leads to an amino acid
substitution. Nucleotide substitutions that do not lead to amino acid
substitutions are referred to as {\it synonymous substitutions},
because the codons involved are synonymous, i.e., code for the same
amino acid. Nucleotide substitutions that do lead to amino acid
substituions are {\it non-synonymous substitutions}.\index{genetic code!redundancy}\index{synonymous substitutions}\index{non-synonymous substitutions}

\begin{table}
\begin{center}
\begin{tabular}{lll}
\hline\hline
      & Amino & \\
Codon & Acid  & Redundancy \\
\hline
CCU   & Pro   & 4-fold \\
CCC \\
CCA \\
CCG \\
\hline
AAU   & Asn   & 2-fold \\
AAC \\
AAA   & Lys   & 2-fold \\
AAG \\
\hline
\end{tabular}
\end{center}
\caption{Examples of 4-fold and 2-fold redundancy in the 3rd position
  of the universal genetic code.}\label{table:fold}
\end{table}

\section*{Rates of synonymous and non-synonymous substitution}

By using a modification of the simple Jukes-Cantor model we
encountered before, it is possible to make separate estimates of the
number of synonymous substitutions and of the number of non-synonymous
substitutions that have occurred since two sequences diverged from a
common ancestor. If we combine an estimate of the {\it number\/} of
differences with an estimate of the {\it time of divergence\/} we can
estimate the rates of synonymous and non-synonymous
substitution~(number/time). Table~\ref{table:substitution-data} shows
some representative estimates for the rates of synonymous and
non-synonymous substitution in different genes studied in
mammals.\index{substitution rate}

\begin{table}
\begin{center}
\begin{tabular}{lcc}
\hline\hline
Locus     & Non-synonymous rate & Synonymous rate \\
\hline
Histone \\
\quad H4  & 0.00                & 3.94 \\
\quad H2  & 0.00                & 4.52 \\
Ribosomal proteins \\
\quad S17 & 0.06                & 2.69 \\
\quad S14 & 0.02                & 2.16 \\
Hemoglobins \& myoglobin \\
\quad $\alpha$-globin & 0.56    & 4.38 \\
\quad $\beta$-globin  & 0.78    & 2.58 \\
\quad Myoglobin       & 0.57    & 4.10 \\
Interferons \\
\quad $\gamma$  & 3.06          & 5.50 \\
\quad $\alpha$1 & 1.47          & 3.24 \\
\quad $\beta$1  & 2.38          & 5.33 \\
\hline
\end{tabular}
\end{center}
\caption{Representative rates of synonymous and non-synonymous
  substitution in mammalian genes~(from~\cite{Li97}). Rates are
  expressed as the number of substitutions per $10^9$
  years.}\label{table:substitution-data}
\end{table}

Two very important observations emerge after you've looked at this
table for awhile. The first won't come as any shock. The rate of
non-synonymous substitution is generally lower than the rate of
synonymous substitution. This is a result of the ``sledgehammer
principle'' I mentioned earlier. Just as taking a sledgehammer to your
car engine and making random changes is unlikely to make it run
better, so making random changes to the amino acid composition of a
protein is unlikely to make it function better. Mutations that change
the amino acid sequence of a protein are more likely to reduce that
protein's functionality than to increase it. As a result, they are
likely to lower the fitness of individuals carrying them, and they
will have a lower probability of being fixed than those mutations that
do not change the amino acid sequence.\index{sledgehammer principle}\footnote{Remember 
our discussion of the probability that disfavored mutations are fixed
as a result of natural selection. They can be fixed, but they are less
likely to be fixed than those that are neutral.}
  
The second observation is more subtle. Rates of non-synonymous
substitution vary by more than two orders of magnitude: 0.02
substitutions per nucleotide per billion years in ribosomal protein
S14 to 3.06 substitutions per nucleotide per billion years in
$\gamma$-interferon, while rates of synonymous substitution vary only
by a factor of two (2.16 in ribosomal protein S14 to 5.50 in $\gamma$
interferons. If synonymous substitutions are neutral, as they probably
are to a first approximation,\footnote{We'll see that they may not be
  completely neutral a little later, but at least it's reasonable to
  believe that the intensity of selection to which they are subject is
  a lot less than that to which non-synonymous substitutions are
  subject.}  then the rate of synonymous substitution should equal the
mutation rate. Thus, the rate of synonymous substitution should be
approximately the same at every locus, which is roughly what we
observe. But proteins differ in the degree to which their
physiological function affects the performance and fitness of the
organisms that carry them. Some, like histones and ribosomal proteins,
are intimately involved with chromatin or translation of messenger RNA
into protein. It's easy to imagine that just about any change in the
amino acid sequence of such proteins will have a detrimental effect on
their function. Others, like interferons, are involved in responses to
viral or bacterial pathogens. It's easy to imagine not only that the
selection on these proteins might be less intense, but that some amino
acid substitutions might actually be favored by natural selection
because they enhance resistance to certain strains of pathogens. Thus,
the probability that a non-synonymous substitution will be fixed is
likely to vary substantially among genes, just as we observe.

\section*{Revising the neutral theory}

So we've now produced empirical evidence that many mutations are {\it
  not\/} neutral. Does this mean that we throw the neutral theory of
molecular evolution away? Hardly. We need only modify it a little to
accomodate these new observations.\index{neutral theory!modifications}

\begin{itemize}

\item {\it Most non-synonymous substitutions are deleterious.\/} We
  can actually generalize this assertion a bit and say that most
  mutations that affect function are deleterious. After all, organisms
  have been evolving for about 3.5 billion years. Wouldn't you expect
  their cellular machinery to work pretty well by now?

\item {\it Most molecular variability found in natural populations is
    selectively neutral.} If most function-altering mutations are
  deleterious, it follows that we are unlikely to find much variation
  in populations for such mutations. Selection will quickly eliminate
  them.\footnote{Remember, when I say that ``the variability is
    selectively neutral,'' that's shorthand for saying that ``the
    product of effective population size and the selection coefficient
    on different alleles is less than one, meaning that the dynamics
    of allele frequency change are more similar to those of an allele
    that has no effects on fitness than to those of an allele with an
    effect on fitness when we can neglect genetic drift.''}

\item {\it Natural selection is primarily purifying.} Although natural
  selection for variants that improve function is ultimately the
  source of adaptation, even at the molecular level, most of the time
  selection is simply eliminating variants that are less fit than the
  norm, not promoting the fixation of new variants that increase
  fitness.

\item {\it Alleles enhancing fitness are rapidly
    incorporated.}\footnote{To be more precise I should have written
    {\it Alleles enhancing fitness are rapidly incorporated, {\bf when
        they are not lost quickly as a result of genetic drift}.\/}}
  They do not remain polymorphic for long, so we aren't likely to find
  them when they're polymorphic.

\end{itemize}

As we'll see, even these revisions aren't entirely sufficient, but
what we do from here on out is more to provide refinements and
clarifications than to undertake wholesale revisions.

\bibliography{popgen}
\bibliographystyle{plain}

\ccLicense

\end{document}


